\documentclass[UTF8, 11pt]{ctexart}
\usepackage{amsmath, CJKutf8, setspace, tabu, multirow, fancyhdr, indentfirst, geometry, graphicx, color,xcolor, makecell}

\geometry{left=4.0cm,right=4.0cm,top=4.8cm,bottom=4.8cm}

\begin{document}

\pagestyle{fancy}
\lhead{正常的NOIP模拟赛}
\rhead{xcy}
\renewcommand{\headrulewidth}{0.4pt}
\date{}
\title{正常的NOIP模拟赛}
\author{xcy}

\begin{CJK}{UTF8}{gkai}
\begin{spacing}{1.2}
    \maketitle
    \begin{center}
    \begin{tabular}{|p{4cm}|p{3cm}|p{3cm}|p{3cm}|}
        \hline
        题目名称 & a & b & c \\
		\hline
        输入文件名 & a.in & b.in & c.in \\
        \hline
        输出文件名 & a.out & b.out & c.out \\
        \hline
        题目类型 & 传统型 & 传统型 & 传统型 \\
        \hline
        时间限制 & 1s & 1s & 2s \\
        \hline
        空间限制 & 512MB & 512MB & 512MB \\
        \hline
        代码长度限制 & 50KB & 50KB & 50KB \\
		\hline
        测试点个数 & 20 & 20 & 20 \\
		\hline
		每个测试点分值 & 5 & 5 & 5 \\
        \hline
	\end{tabular}
	\end{center}

	注意: 


	\ \ \ \ 1. 第$2$题和第$3$题有大样例, 不同的大样例对应不同的部分分.
	
	\ \ \ \ \ \ \ 大样例在/down/samples中.

	\ \ \ \ 2. 发现原题请勿声张.

	\ \ \ \ 3. 评测时开O2, 但不开C++11.

	\newpage
	\section{a}
		\subsection{Description}
			给定$T\in\{1,2\}$, 再给定正整数$N$.

			若$T=1$, 输出二元方程$\frac{1}{x}+\frac{1}{y}=\frac{1}{N}$的正整数解$(x,y)$的个数.

			若$T=2$, 输出二元方程$\frac{1}{x}+\frac{1}{y}=\frac{1}{N!}$的正整数解$(x,y)$的个数.

			答案均对$10^9+7$取模.
		\subsection{Input Format}
			读入一行两个数$T,N$.
		\subsection{Output Format}
			输出一行一个数, 表示答案.
		\subsection{Sample 1}
		\subsubsection{Input}
		\begin{verbatim}
		    1 120
		\end{verbatim}
		\subsubsection{Output}
		\begin{verbatim}
		    63
		\end{verbatim}
		\subsection{Sample 2}
		\subsubsection{Input}
		\begin{verbatim}
		    2 1439
		\end{verbatim}
		\subsubsection{Output}
		\begin{verbatim}
		    102426508
		\end{verbatim}
		\subsection{Constraints}
			\begin{center}
			\begin{tabular}{p{2cm}<{\centering}|p{1.5cm}<{\centering}|p{1.5cm}<{\centering}|p{1.5cm}<{\centering}}
			\Xhline{1.5pt}
			测试点编号 & $T$ & $N$ & 分值\\
			\Xhline{1.2pt}
			1$\sim$4 & \multirow{2}*{=1} & $\le 10^6$ & 20 \\
			\cline{1-1}
			\cline{3-4}
			5$\sim$8 & & $\le 10^{12}$ & 20\\
			\hline
			9$\sim$12 & \multirow{2}*{=2} & $\le 15$ & 20 \\
			\cline{1-1}
			\cline{3-4}
			13$\sim$20 & & $\le 10^{6}$ & 40\\
			\Xhline{1.5pt}
			\end{tabular}
			\end{center}

	\newpage
	\section{b}
		\subsection{Description}
			$N$个机器人排布在数轴上. 给定每个机器人所在点的坐标$x_i$, 交流半径$r_i$和兼容性$q_i$. 处在$x_i$的机器人, 其交流范围为$[x_i-r_i, x_i+r_i]$.

			两个机器人$i,j$能进行交流当且仅当每个机器人都在对方的交流范围内(即$x_i\in[x_j-r_j,x_j+r_j]$且$x_j\in[x_i-r_i,x_i+r_i]$), 且它们的兼容性之差的绝对值不超过一个给定的常数$K$(即$|q_i-q_j|\le K$).

			问有多少对机器人能进行交流. $(i,j)$和$(j,i)$算作同一对.
		\subsection{Input Format}
			第一行有两个数$N,K$.

			第$2\sim (N+1)$行, 每行三个数, 第$i$行为$x_{i-1},r_{i-1},q_{i-1}$.
		\subsection{Output Format}
			输出一行一个数, 表示答案.
		\subsection{Sample}
		\subsubsection{Input}
		\begin{verbatim}
		    3 2
		    3 6 1
		    7 3 10
		    10 5 8
		\end{verbatim}
		\subsubsection{Output}
		\begin{verbatim}
		    1
		\end{verbatim}
		\subsubsection{Explanation}
			只有第$2$个和第$3$个机器人才能进行交流.
		\subsection{Constraints}
			Subtask 1: $N\le10^3$, 30pts.

			Subtask 2: $K=0$, 20pts.

			Subtask 3: 无特殊限制, 50pts.

			对于所有的数据, $1\le N\le10^5$, $0\le K \le 20$, $0\le x_i,r_i\le10^9$, $0\le q_i\le 5\times 10^8$.
	\newpage
	\section{c}
		\subsection{Description}
			给定一棵$n$个点的无根树, 其中点$a$, $b$是黑色的, 其他点是白色的. $a$, $b$可能是同一个点, 这样相当于只有那一个点是黑色的, 其他$(n-1)$个点是白色的.

			接下来不断进行操作, 每次操作把一个黑点染红, 同时把和它相邻的白点染黑. 所有点均被染红时操作完成.

			问把所有点染红有多少种操作方案. 两方案不同当且仅当依次被操作的点的形成的序列不同.

			答案对$998244353$取模.
		\subsection{Input Format}
			第一行三个数, $n$, $a$, $b$.

			接下来$(n-1)$行, 每行两个数$x$, $y$, 表示树上的一条边$(x,y)$.
		\subsection{Output Format}
			输出一行一个数, 表示答案.
		\subsection{Sample}
		\subsubsection{Input}
		\begin{verbatim}
		    4 1 2
		    1 2
		    2 3
		    3 4
		\end{verbatim}
		\subsubsection{Output}
		\begin{verbatim}
		    4
		\end{verbatim}

		\subsection{Constraints}
			对于$20\%$的数据, $n\le 10$.

			对于另外$20\%$的数据, $a=b$.

			对于另外$20\%$的数据, 每个点的度数不超过$2$.

			对于$100\%$的数据, $n\le5000$.
		\newpage

\end{spacing}
\end{CJK}

\end{document}
