\documentclass[12pt, a4paper]{article}
\usepackage[slantfont, boldfont]{xeCJK}
\usepackage{ulem}
\usepackage{amsmath}
\usepackage{booktabs}
\usepackage{colortbl}
\usepackage{amsmath}
\usepackage{algorithm}
\usepackage{algpseudocode}
\usepackage{amsmath}
\renewcommand{\algorithmicrequire}{\textbf{Input:}}  % Use Input in the format of Algorithm
\renewcommand{\algorithmicensure}{\textbf{Output:}} % Use Output in the format of Algorithm
\usepackage{indentfirst}
\usepackage[top = 1.0in, bottom = 1.0in, left = 1.0in, right = 1.0in]{geometry}
\usepackage{listings}
\usepackage{enumerate}
\usepackage{multirow}
\usepackage[usenames,dvipsnames]{xcolor}
\usepackage{graphicx}

\setCJKmainfont{SimSun}
\setCJKmonofont{SimSun}

\setlength{\parskip}{0.6\baselineskip} %行距
\setlength{\parindent}{2em} %缩进

\newcolumntype{Y}{>{\columncolor{red}}p{12pt}}
\newcolumntype{N}{>{\columncolor{white}}p{12pt}}

\renewcommand\arraystretch{1.2} %表格行高

\title{模拟赛}
\begin{document}
\maketitle
\begin{center}
\begin{tabular}{|p{90pt}|p{90pt}|p{90pt}|p{90pt}|}
	\hline
	题目名称 & 序列 & 进制 & 矩阵 \\
	\hline
	目录 & seq & hex & matrix \\
	\hline
	可执行文件名 & seq & hex & matrix \\
	\hline
	输入文件名 & seq.in & hex.in & matrix.in \\
	\hline
	输出文件名 & seq.out & hex.out & matrix.out \\
	\hline
	每个测试点时限 & $1.0s$ & $1.0s$ & $2.0s$ \\
	\hline
	内存限制 & 512MB & 512MB & 512MB \\
	\hline
	试题总分 & 100 & 100 & 100 \\
	\hline
	测试点数目 & 10 & 20 & 20 \\
	\hline
	每个测试点分值 & 10 & 5 & 5 \\
	\hline
	是否有部分分 & 否 & 否 & 否\\
	\hline
	题目类型 & 传统型 & 传统型 & 传统型\\
     \hline
\end{tabular}

提交的源程序文件名

\begin{tabular}{|p{90pt}|p{90pt}|p{90pt}|p{90pt}|}
	\hline
	对于 C++ 语言 & seq.cpp & hex.cpp & matrix.cpp \\
	\hline
	对于 C 语言 & seq.c & hex.c & matrix.c \\
	\hline
	对于 Pascal 语言 & seq.pas & hex.pas & matrix.pas \\
	\hline
\end{tabular}

编译开关

\begin{tabular}{|p{90pt}|p{90pt}|p{90pt}|p{90pt}|}
	\hline
	对于 C++ 语言 & -O2 -std=c++11 & -O2 -std=c++11 & -O2 -std=c++11 \\
	\hline
	对于 C 语言 & -O2 -std=c11 & -O2 -std=c11 & -O2 -std=c11 \\
	\hline
	对于 Pascal 语言 & -O2 & -O2& -O2 \\
	\hline
\end{tabular}

\end{center}
\begin{center}
\end{center}


\newpage
\section{序列}

\subsection{题目描述}

给定一个长度为偶数的排列$p$,以及一个初始时为空的序列$q$,对其进行如下操作直到$p$为空:

从$p$中找出两个相邻元素,按原来在$p$中的顺序加入$q$头部,然后把它们从$p$中删去。

求可能得到的字典序最小的$q$。

\subsection{输入格式}

从文件seq.in中读取数据。

第一行一个整数 $N$ ,表示排列$p$长度。

接下来一行$N$个整数,其中第$i$个整数$p_i$表示排列$p$的第$i$个元素。

\subsection{输出格式}

输出到文件seq.out中。

输出一行$N$个整数,表示字典序最小的$q$。

\subsection{样例1输入}
\begin{quote}
\begin{verbatim}
4
3 2 4 1
\end{verbatim}
\end{quote}
\subsection{样例1输出}
\begin{quote}
\begin{verbatim}
3 1 2 4
\end{verbatim}
\end{quote}
\subsection{样例2输入}
\begin{quote}
\begin{verbatim}
8
4 6 3 2 8 5 7 1
\end{verbatim}
\end{quote}
\subsection{样例2输出}
\begin{quote}
\begin{verbatim}
3 1 2 7 4 6 8 5
\end{verbatim}
\end{quote}

\subsection{数据范围与约定}

对于$20\%$的数据,$1 \leq N \leq 10$;

对于$40\%$的数据,$1 \leq N \leq 2000$;

对于额外$10\%$的数据,保证$p_i=i$;

对于$100\%$的数据,保证$1 \leq N \leq 2\times 10^5$。

\newpage
\section{进制}

\subsection{题目描述}

将所有正整数的十六进制从小到大依次写下,会形成一个长度无穷的序列,它的前几位是123456789ABCDEF1011121314…。

依次回答$T$组询问,分为如下两种:

1.给定正整数$n$,回答这个序列第$n$位上的字符;

2.给定正整数$n$以及字符$c$,回答序列的前$n$位中,有多少位是$c$,保证$c$是$0,1,2,3,4,5,6,7,8,9,A,B,C,D,E,F$中的一个。

\subsection{输入格式}

从文件hex.in中读取数据。

第一行一个整数$T$,表示数据组数。

接下来$T$行,每行先读入一个整数$typ$表示数据类型。

如果$typ=1$,接下来读入一个正整数$n$,然后回答一组询问1;

如果$typ=2$,接下来读入一个正整数$n$和一个字符$c$,然后回答一组询问2。

\subsection{输出格式}

输出到文件hex.out中。

对于每组数据,输出一行一个字符或一个整数表示答案。

\subsection{样例输入}
\begin{quote}
\begin{verbatim}
10
1 20
1 21
1 22
2 21 0
2 21 1
2 21 2
2 200 2
2 2000 0
2 2000 1
2 2000 2
\end{verbatim}
\end{quote}
\subsection{样例输出}
\begin{quote}
\begin{verbatim}
1
2
1
1
5
2
23
79
352
342
\end{verbatim}
\end{quote}
\subsection{数据范围与约定}

对于$15\%$的数据,$1 \leq T \leq 20$,$1 \leq N \leq 2000$;

对于$30\%$的数据,$1 \leq N \leq 10^6$;

对于额外$20\%$的数据,保证$typ=1$;

对于$100\%$的数据,保证$1 \leq T \leq 10^5$,$typ \in \{1,2\}$,$1 \leq N \leq 10^{18}$。

\newpage
\section{矩阵}
\subsection{题目描述}

有一个$N$行$M$列的矩阵,你可以选择一些位置涂黑,其它位置涂白。

对一个矩阵,我们计算出一个长度为$N$的序列$A$,以及两个长度为$M$的序列$B$和$C$:

令$A_i$为第$i$行第一次出现黑色的位置的列号,如果第$i$行全白,则为$M+1$;

令$B_i$为第$i$列第一次出现黑色的位置的行号,如果第$i$列全白,则为$N+1$;

令$C_i$为第$i$列最后一次出现黑色的位置的行号,如果第$i$列全白,则为$0$;

请你计算,有多少种不同的$\{A,B,C\}$三元组可能得到,输出答案对$998244353$取模后的结果。

\subsection{输入格式}

从文件matrix.in中读取数据。

输入共一行两个整数$N$,$M$,分别表示矩阵的行数和列数。

\subsection{输出格式}

输出到文件matrix.out中。

输出一行一个整数,表示答案。

\subsection{样例输入1}
\begin{quote}
\begin{verbatim}
2 3
\end{verbatim}
\end{quote}
\subsection{样例输出1}
\begin{quote}
\begin{verbatim}
64
\end{verbatim}
\end{quote}
\subsection{样例输入2}
\begin{quote}
\begin{verbatim}
4 3
\end{verbatim}
\end{quote}
\subsection{样例输出1}
\begin{quote}
\begin{verbatim}
2588
\end{verbatim}
\end{quote}
\subsection{数据范围与约定}

对于$20\%$的数据,$1 \leq N \leq 6$,$1 \leq M \leq 3$;

对于$50\%$的数据,$1 \leq N \leq 200$,$1 \leq M \leq 200$;

对于额外$5\%$的数据,保证$M=1$;

对于额外$15\%$的数据,$1 \leq N \leq 2000$,$1 \leq M \leq 3$;

对于$100\%$的数据,保证$1 \leq N \leq 8000$,$1 \leq M \leq 200$。

\end{document}