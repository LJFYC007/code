\documentclass{beamer}

\usepackage{xeCJK}

\setCJKmainfont[BoldFont = {黑体}]{宋体}
\setlength{\parindent}{22pt}

\usetheme{CambridgeUS}
\usecolortheme{dolphin}
\usefonttheme{serif}

\title{Day1-题解}
\author{}
\institute{}

\begin{document}

\begin{frame}
	\maketitle
\end{frame}

\section{T1}
\begin{frame}{小D的奶牛}

	\par 题意:给一个$N$个点的图,问图中有多少个团;

	\par $N \leq 50$

\end{frame}

\begin{frame}{测试点1-5}

	\par 测试点1: 暴力;

	\pause

	\par 测试点2,3:状压dp;

	\pause

	\par 测试点4,5:团的大小$\leq 21$:

	\par 枚举团上编号最小的点,然后同上一部分;

\end{frame}

\begin{frame}{测试点6-10}

	\par 测试点6-10:考虑Meet in the middle:

	\pause 

	\par 先考虑前$N / 2$个点,求出每种取点方式是否能形成一个团;

	\par 再考虑后$N / 2$个点,求出每种取点方式有多少个子集是一个团;

	\pause

	\par 枚举每个前部分的点的选取方式,然后求出后半部分有哪些点和它们都有边;

	\par 然后后半部分能取的集合的数量就是,后半部分那些点的子集中形成的团的数量;

	\pause

	\par $O(N * 2 ^ (N / 2))$

\end{frame}

\section{T2}

\begin{frame}{小D的交通}

	\par 题意:问是否存在长度为$N$的连续整数数列,在不互质的数连边后构成的图联通;

	\par $N \leq 100000$

\end{frame}

\begin{frame}{测试点1-优秀的观察}

	\par 当$N$比较小的时候应该无解。

	\par 经过计算(暴力)得出,$N \leq 16$时无解;

	\pause

	\par 所以输出"No solution"可以通过测试点1;

\end{frame}

\begin{frame}{测试点2-3-爆搜}

	\par 有且只有$< N$的质数是有用的;

	\pause

	\par 爆搜序列的开头的数$A$在这些质数下的余数分别是多少;		

	\pause

	\par 而当一个质数$ > N / 2$时,它只能在图中贡献一条边;

	\pause

	\par 又由于所有$\%2 = 0$的数肯定连在一起;

	\par 所以他只能连通一个点,对于后半部分质数,枚举那个点就行了;

\end{frame}

\begin{frame}{测试点4-5-爆搜剪枝}

	\par 这一部分比较玄学;

	\pause

	\par 由于所有偶数点是通过$2$互相连通的,只要考虑奇数点如何连向偶数点;

	\par 不难发现,当质数$> N / 2$时,后面的质数越大越没用,可以贪心连边;

	\pause

	\par 同时,经过最开始的一些比较小的质数之后,大部分奇数点已经和偶数点连通了;

	\par 所以对于那些不大不小的质数,其实决策数并不多,且尽量在枚举时优先枚举贡献较大的连法;

	\pause

	\par 视实现优劣可以得到$30-50+$分;

\end{frame}

\begin{frame}{测试点6-10-一个小清新构造}

	\par 写过暴力的都知道,当$N$不小时,它跑得很快,且总是有解(?);

	\pause

	\par 所以我们尝试构造一组解;

	\pause

	\par 如果我们构造一个以序列开头为核心的解,即在它这个位置的数是所有质数的倍数;

	\pause

	\par 我们发现除了第二个节点之外的点都连通了;

	\pause

	\par 如果我们构造一个以序列中心为核心的解,那么接下来要解决的问题就是和它相邻的点;

	\par 我们用最大的小于$N / 2$的质数来使这两个点与其他点连通;

	\pause

	\par 但这样由于核心少用了两个数,可能会导致有一些其他的点不连通;

	\pause

	\par 不要慌,它们都在最边上,且大于$N / 2$的质数我们还没有使用,随便连一连就好;

	\pause

	\par 所以我们就知道了开头模每个质数的余数,最后用中国剩余定理还原答案即可;

\end{frame}

\section{T3}
\begin{frame}{小D的远航}

	\par 题意:一个对称的凸的连通块,求它整体移出迷宫的最小步数;

	\par $N \leq 2000$

\end{frame}

\begin{frame}{测试点1-4}

	\par $N \leq 400$;

	\pause

	\par $O(N)$判断一个位置是否能放下船;

	\pause

	\par bfs求出出地图的最短路;

	\par 总复杂度$O(N^3)$

\end{frame}

\begin{frame}{测试点5-8}

	\par 船是矩形或菱形;

	\pause

	\par 二维前缀和,$O(N^2)$;

\end{frame}

\begin{frame}{测试点9-12+}

	\par 数据比较水(虽然所有的点好像都挺水的);

	\pause

	\par 枚举边界线之类的乱搞;

\end{frame}

\begin{frame}{测试点13-20}

	\par 为了方便描述,在接下来的讨论中,我们认为船的对称轴(记作L)平行于Y轴;

	\pause

	\par 障碍我们一行一行讨论,对于每一行,我们把所有的障碍按顺序排好。

	\pause

	\par 接下来我们一列一列考虑有哪些位置可以放船,再用每一行的障碍去限制;

	\pause

	\par 假设考虑到第$i$列,对船的限制最紧的应该是最靠近第$i$列的障碍;

	\par 这个可以从船的凸性中轻松推出;

	\pause

	\par 所以我们只要考虑$O(N)$个障碍对这一列的影响;

	\par 求出这些障碍后,就可以$O(N)$时间内得到一列的所有的能放船的位置;

	\pause

	\par 关于求靠这一列最近的障碍,在枚举列的时候顺便更新一下;

	\pause

	\par 总的复杂度是$O(N^2)$的,细节比较复杂;

\end{frame}

\end{document}