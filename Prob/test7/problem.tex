 
\documentclass[UTF8]{ctexart}
\usepackage{booktabs}
\usepackage{amsmath}
\usepackage{geometry}
\usepackage{fancyhdr}
\usepackage{listings}
\usepackage{color}
\usepackage{cancel}
\usepackage{ulem}

\begin{document}

\begin{center}
    \large{小Li的宝藏}
\end{center}
\paragraph{题目描述} QwQ

小Li又想去寻宝了,这次他找到了\sout{某校}下面的一个古墓,很神奇的是,这个古墓是一棵\textbf{树}。这个古墓里每一个点都有一个机关,机关关闭时表示当前这个点可以经过,开启是表示当前这个点不可以经过

古墓是动态的,在某些时刻会改变一个点的机关状态(即开变关,关变开),同时小Li害怕自己会无法从古墓中离开,所以会在某些时刻询问从时刻1至当前时刻有多少时刻满足路径 $u$ 至 $v$ 中所有点开关均关闭

具体的,输入有两种

\begin{itemize}

\item $1\ x\ t$ 表示当前为时刻 $t$ 进行修改操作,将 $x$ 点的机关状态取反
\item $2\ u\ v\ t$ 表示当前为时刻 $t$ 进行查询操作,具体如上

\end{itemize} 

树的定义如下:

\paragraph{输入格式}
\paragraph{}
第一行两个数字$n,q$表示树的节点个数与操作个数

接下来的 $n-1$ 行,每行包含两个正整数 $u$ 和 $v$ ($1\le u,v\le n$),表示 $u$ 和 $v$ 之间由一条边相连。

第$n+1$行,有$n$个数字$a_1,a_2,a_3\ldots a_n$,表示$n$个点的点权($|a_i|\leq 10^9$)

接下来的 $q$ 行,每行的最开始包含一个整数$opt$

若$opt=1$ 接下来有两个数$x,y$,表示将$x$号点的点权修改成$y$

否则 接下来有一个数$x$,表示询问以$x$为根的带边权点权和(对$10^9+7$取模)

数据保证给出的边能构成一棵树。

\paragraph{输出格式}
\paragraph{}对于每个操作2输出一行答案在$mod$$\,{10^9+7}$意义下的最小正整数
\paragraph{样例1输入}
\begin{lstlisting}
    10 4
    1 2
    1 3
    2 6
    2 4
    4 8
    8 9
    3 5
    3 7
    5 10
    2 1 3 4 2 3 4 2 1 -1
    2 1
    1 8 5
    2 1
    2 6
\end{lstlisting}
\paragraph{样例1输出}
\begin{lstlisting}
    59
    77
    122
\end{lstlisting}
\paragraph{数据范围}
\paragraph{}Subtask1.对于$20\%$的数据,$n,q\leq 2000$
\paragraph{}Subtask2.对于$40\%$的数据,$n,q\leq 50000$
\paragraph{}Subtask3.对于另外$10\%$的数据,保证图的形态为菊花
\paragraph{}Subtask4.对于另外$10\%$的数据,保证图的形态为链
\paragraph{}Subtask5.对于另外$10\%$的数据,数据随机
\paragraph{}对于$100\%$的数据,$n,q\leq 3*10^5$
\paragraph{提示}
\paragraph{}题目其实很\textbf{套路}

\end{document}