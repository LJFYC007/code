\documentclass[9pt]{beamer}
\usepackage{xeCJK}
\setsansfont{Monaco}
\setCJKmonofont{AR PL UKai CN}
\usetheme{metropolis}
\title{极其简单的DP选讲}
\date{\today}
\author{ljfcnyali}
\begin{document}

  \maketitle
  \begin{frame}{目录}
    \par 一些DP技巧
    \par 插头DP
    \par 动态DP
  \end{frame}

  \section{一些DP技巧}
  \begin{frame}{[SDOI2011]拦截导弹}
    \par 有$n$个导弹,每个导弹有高度$h$和速度$v$,拦截一个导弹的要求是高度和速度不能增加,求最多能拦截多少导弹
    \par $n\le 5\times 10^4,h_i,v_i\le 1000$
  \end{frame}
  
  \begin{frame}{[SDOI2011]拦截导弹}
    \onslide<1-> 这道题DP方程特别简单,设$dp_i$表示拦截第$i$个导弹时最多可以拦截多少导弹,有方程$dp_i=max_{1\le j\le i-1}(dp_j+1)$
    
    \onslide<2-> 主要考虑优化状态转移,有一种很巧妙的方式是使用CDQ分治(还可以用KD-Tree?)

    \onslide<3-> 其实在CDQ分治里只需要改变CDQ分治的顺序,先递归计算$l-Mid$,再计算左半对右半的更新,最后递归计算$Mid+1-r$
    
    \onslide<4-> 为什么是对的?

    \onslide<5-> DP转移需要满足转移必须有序,即不可以出现半成品,可以发现这种顺序相当于CDQ分治树的中序遍历

    \onslide<6-> 这题其实还有一个第二问,但是与我们的标题不符就删去了
  \end{frame}

  \begin{frame}{斜率优化 [APIO2010]特别行动队}
    \onslide<1-> 因为大家学习过斜率优化DP,这里就只讨论一种基本解法

    \onslide<2-> 首先看一道例题

    \onslide<3-> 有一支由$n$个士兵组成的部队,要求分成连续的若干段,每一段的战斗力为$ax^2+bx+c$,其中$x$是这一段士兵的战斗力和,求出最大的战斗力总和
    
    \onslide<4-> $a<0,n\le 10^6$
  \end{frame}

  \begin{frame}{斜率优化 [APIO2010]特别行动队}
    \onslide<1-> 设$sum_i$为前缀和,可以很简单的写出DP方程

    \onslide<2-> $dp_i=max_{0\le j<i}\{dp_j+a\times(sum_i-sum_j)^2+b\times(sum_i-sum_j)+c\}$

    \onslide<3-> 进行斜率优化推式子的基本步骤是假设由$j$转移比由$k$转移更优($k<j$,因为我们维护的是一个单调队列同时应弹出队首),则有:

    \onslide<4-> $dp_j-2asum_isum_j+asum_j^2-bsum_j>dp_k-2asum_isum_k+asum_k^2-bsum_k$

    \onslide<5-> 接下来一步是把同时和$i,j$或$i,k$有关的项移到一边

    \onslide<6-> $dp_j+asum_j^2-bsum_j-dp_k-asum_k^2+bsum_k>2asum_i(sum_j-sum_k)$
  \end{frame}

  \begin{frame}{斜率优化 [APIO2010]特别行动队}
    \onslide<1-> 再将这一边的项中与$j,k$有关的除过去

    \onslide<2-> $$\frac{dp_j+asum_j^2-bsum_j-dp_k-asum_k^2+bsum_k}{sum_j-sum_k}>2asum_i$$
  
    \onslide<3-> 注意到题目中保证$a<0$,则等式右边单调递减并一直为负,所以我们需要维护一个上凸壳

    \onslide<4-> 弹队尾的时候应取小于号(上凸壳是小于号,下凸壳反之,等于号无所谓)
  \end{frame}

  \begin{frame}{四边形不等式优化}
    \onslide<1-> 假设DP式为$f_{i,j}=min_{k=i}^{j-1}\{f_{i,k}+f_{k+1,j}\}+w_{i,j}$时,若满足四边形不等式时可以进行优化:

    \onslide<2-> 四边形不等式:对于$a\le b\le c \le d$均有$w_{a,c}+w_{b,d}\le w_{a,b}+w_{c,d}$,若$w_{i,j}$满足四边形不等式,则$f_{i,j}$也满足四边形不等式

    \onslide<3-> 假设$f_{i,j}$是由$s_{i,j}$转移得到(最优决策点),那么$f_{i,j}$的最优决策点当且仅当在$[s_{i,j-1},s_{i+1,j}]$范围内

    \onslide<4-> 可以证明时间复杂度是$O(n^2)$
  \end{frame}

  \begin{frame}{四边形不等式优化 [IOI2000]邮局}
    \onslide<1-> 有$V$个村庄,给定村庄坐标,需要建立$P$个邮局,使得村庄到其最近的邮局的距离和最小

    \onslide<2-> $P\le 300,V\le 3000$
  \end{frame}

  \begin{frame}{四边形不等式优化 [IOI2000]邮局}
    \onslide<1-> 朴素做法很好想,设$f_{i,j}$表示前$i$个村庄建立$j$个邮局的最小距离和,那么有DP方程

    \onslide<2-> $f_{i,j}=min_{0\le k<i}\{f_{k,j-1}+dis(k+1,i)\}$

    \onslide<3-> 其中$dis(l,r)$表示在村庄$[l,r]$中建立一座邮局的最小距离和,众所周知建立在中位数最优秀,可以进行预处理

    \onslide<4-> 发现$dis_{i,j}$和$f_{i,j}$均满足四边形不等式,可以直接优化,注意$i$需要倒序枚举

    \onslide<5-> 这题还有一种非常巧妙的办法....去看洛谷题解吧
  \end{frame}

  \begin{frame}{四边形不等式优化 决策单调性}
    \onslide<1-> 假设DP式为$f_i=min_{j=1}^{i-1}\{f_j+w(i,j)\}$,并且$w(i,j)$满足四边形不等式,则可以进行优化
    
    \onslide<2-> 这种式子经过证明最优决策点会单调递增,所以可以使用分治的方式得到答案
    
    \onslide<3-> 分治步骤大概是暴力枚举$Mid$的最优决策点,再分为$l,Mid$和$Mid+1,r$求解,这样会不断缩小暴力求解最优决策点的范围,可以证明时间复杂度为$O(nlog_2n)$
  \end{frame}

  \begin{frame}{四边形不等式优化 [POI2011]Lightning Conductor}
    \onslide<1-> 给定$a$,求解所有整数$p_i=max_{j=1}^n\{a_j+\sqrt{|i-j|}-a_i\}$

    \onslide<2-> $n\le 5\times 10^5,a_i\le 10^9$

    \onslide<3-> 为了方便大家思考,给出一个输入输出和每个点的最优决策点

    \onslide<4-> 8 2 3 2 4 2 3 2 4,第一个数为$n$,接下来$n$个数表示$a_i$
    
    \onslide<5-> 5 4 5 2 4 3 4 3,表示$p_i$

    \onslide<6-> 8 8 8 8 8 4 4 4,表示每个点的最优决策点

    \onslide<7-> 0 0 2 2 4 4 4 4,表示只考虑$j<i$的情况的最优决策点

    \onslide<8-> 8 8 8 8 8 8 8 4,表示值考虑$j\ge i$的情况的最优决策点
  \end{frame}

  \begin{frame}{四边形不等式优化 [POI2011]Lightning Conductor}
    \onslide<1-> 观察到这个绝对值不太方便,把绝对值拆开分两种情况
    
    \onslide<2-> 打表会发现拆开后的$p_i$满足决策单调性,其中$j<i$的情况下$p_i$单调递增,反之单调递减

    \onslide<3-> 直接照上面的分治方式求解即可

    \onslide<4-> 总结一下这些优化决策区间的题有两种情况,有决策上下界或单调性,具体最好打表来判断,但解法都是类似的
  \end{frame}

  \begin{frame}{四边形不等式优化 [CF321E]Ciel and Gondolas [BZOJ5311]贞鱼}
    \onslide<1-> 将长度为$n$的序列分为$k$段,当$i,j$在同一段中,会产生$a_{i,j}$的代价,求最小总代价
    
    \onslide<2-> $n\leq 4000,k\leq 800$
  \end{frame}
    
  \begin{frame}{四边形不等式优化 [CF321E]Ciel and Gondolas [BZOJ5311]贞鱼}
    \onslide<1-> 设$dp_{i,j}$表示前$i$个数分为$j$段的最小总代价,有方程
    \onslide<2-> $$dp_{i,j}=max_{0\leq k<i}\{dp_{k,j-1}+sum(k+1,i)\}$$
    
    \onslide<3-> $sum_{i,j}$可以用前缀和直接求出,同时可以发现在$j$固定的情况下,$dp_{i,j}$满足决策单调性,其中
    
    \onslide<4-> 可以使用上面的分治方式,时间复杂度$O(nklog_2n)$,CF可以过,但BZOJ上TLE了

    \onslide<5-> 有一种优化叫wqs优化,可以把时间优化成$O(nlog_2klog_2n)$,但是我不会....
  \end{frame}

  \section{插头DP}

  \begin{frame}{插头DP [HNOI2004]邮递员}
    \onslide<1-> 给定$n\times m$的矩阵,求经过所有点的闭合回路个数

    \onslide<2-> $m\le 10,n\le 20$

    \onslide<3-> 插头DP模板题,答案很大注意开$\_\_int128$
  \end{frame}
  
  \begin{frame}{插头DP [UVA10572]Black\&White}
    \onslide<1-> 一个棋盘,已有部分被染色为黑或白,给所有未染色的格子染上黑或白,使得所有黑格子在同一个连通块,所有白格子在同一连通块。并且不存在一个$2\times 2$的方格中的所有格子颜色相同,求染色方案数并输出一种合法方案

    \onslide<2-> $n,m\le 8,T\le 10$
  \end{frame}

  \begin{frame}{插头DP [UVA10572]Black\&White}
    \onslide<1-> 首先考虑输出一种合法方案,只需要记录一下任意一个得到答案的路径即可

    \onslide<2-> 因为不能有$2\times 2$的连通块,所以对于每一个点要记录一下其左上方的点,另外维护的状态应该包含颜色和连通性

    \onslide<3-> 因为这道题不需要插头,所以最小表示法使用记录为第几个连通块,每一次状态转移中连通块个数最多为$m$个

    \onslide<4-> 分类讨论即可(我没调处来...弃疗了)
  \end{frame}

  \section{动态DP}

  \begin{frame}{动态DP}
    \onslide<1-> 看oi-wiki.org,里面的是我写的....
  \end{frame}

  \begin{frame}{练习题}
    \onslide<1-> 删去了几道题,放到后面可以当做练习题做做

    \onslide<2-> [NOI2007]货币兑换:这道题是CDQ分治+斜率优化,算是一开始讲的两个知识点的综合应用

    \onslide<3-> [NOI2014]购票:这道题是树上的斜率优化,因为斜率的等式右边不满足单调性需要二分(或三分)维护,另外有一些距离限制可以用树剖或点分治解决

    \onslide<4-> [NOI2009]诗人小G:这题是四边形不等式的练习题,但是需要二分来维护决策单调性,很有意思

    \onslide<5-> [GDKOI2016]Map:跟上面那道插头DP有点类似,也是维护连通性,可以用来练练手
  \end{frame}

  \section{Thanks}

\end{document}
