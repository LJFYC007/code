\documentclass[UTF8]{ctexart}
\usepackage{graphicx}
\usepackage{booktabs}
\usepackage{listings}
\usepackage{multirow}
\usepackage{color}
\usepackage{mathtools}
\pagestyle{plain}
\begin{document}
\title{弱省省选模拟赛}
\author{}
\date{\today}
\maketitle
\begin{table}[!htbp]
	\centering
	\begin{tabular}{|c|c|c|c|}
		\hline
		题目名称&Easiest&Medium&\\
		\hline
		题目类型&传统型&传统型&传统型\\
		\hline
		目录&easiest&medium&\\
		\hline
		可执行文件名&easiest&medium&\\
		\hline
		输入文件名&easiest.in&medium.in&.in\\
		\hline
		输出文件名&easiest.out&medium.out&.out\\
		\hline
		每个测试点时限&2.0s&2.0s&1.0s\\
		\hline
		内存限制&512MB&512MB&512MB\\
		\hline
		测试点/包数目&6&20&20\\
		\hline
		测试点是否等分&是&是&是\\
		\hline
	\end{tabular}
\end{table}
提交源程序文件名
\begin{table}[!htbp]
	\centering
	\begin{tabular}{|ccc|c|c|c|}
		\hline
		对于&C++&语言&easiest.cpp&medium.cpp&.cpp\\
		\hline
		对于&C&语言&easiest.c&medium.c&.c\\
		\hline
	\end{tabular}
\end{table}
注意事项:
\begin{enumerate}
	\item 文件名(包括程序名和输入输出文件名)必须使用英文小写。
	\item 结果比较方式为忽略行末空格、文末回车后的全文比较。
	\item C/C++ 中函数 main() 的返回值类型必须是 int,值为 0。
	\item 编译选项为-O2 -std=c++11
	\item 如果对题目有疑问(如样例出锅),可以找出题人
	\item \textbf{考试时间8:00至13:00}
\end{enumerate}
\clearpage



\begin{center}
	\large{Easiest}
\end{center}
\paragraph{题目描述}
\paragraph{}给定长度$n$,下标$0\sim n-1$,和两个数组$l_i,r_i$,求$0\sim n-1$的排列$p_i$个数,满足$\forall i,p_i\in[l_i,r_i]$
\paragraph{}满足$n$是$2$的幂,$l,r$单调不递减,且存在二元组$(x,y)$满足$\forall j\in[0,y],j\in[l_x,r_x]$且$\forall j\in[x,n),y\in[l_j,r_j]$
\paragraph{}本题中,认为$[c,c]$为$\{c\}$
\paragraph{输入格式}
\paragraph{}从\emph{easiest.in}中读入数据
\begin{itemize}
	\item 第一行一个整数$n$
	\item 二到三行各$n$个整数,表示$l_i,r_i$
\end{itemize}
\paragraph{输出格式}
\paragraph{}输出到\emph{easiest.out}中
\begin{itemize}
	\item $ans$,即排列数,对$998244353$取模
\end{itemize}
\clearpage
\paragraph{样例1输入}
\begin{lstlisting}
3
0 2
0 2
1 2
\end{lstlisting}
\paragraph{样例1输出}
\begin{lstlisting}
4
\end{lstlisting}
\paragraph{样例1解释}
\paragraph{}$[0,1,2],[0,2,1],[1,0,2],[2,0,1]$
\paragraph{样例2}
\paragraph{}见选手文件夹本题目录下\emph{/2.in}与\emph{/2.out}
\paragraph{样例3}
\paragraph{}见选手文件夹本题目录下\emph{/3.in}与\emph{/3.out}
\paragraph{样例4}
\paragraph{}见选手文件夹本题目录下\emph{/4.in}与\emph{/4.out}
\paragraph{样例5}
\paragraph{}见选手文件夹本题目录下\emph{/5.in}与\emph{/5.out}
\clearpage
\paragraph{子任务}
\paragraph{}所有子任务满足:$n\in[1,300],0\le l_i,r_i< n$
\paragraph{}
\begin{center}
	\begin{tabular}{|c|c|c|c|}
		\hline
		Subtask编号&分值&性质\\
		\hline
		1&1&$l_i=0,r_i=n-1$\\
		\hline
		2&2&$n\le 10$\\
		\hline
		3&4&$l_i=0$或$r_i=n-1$\\
		\hline
		4&8&$n\le 20$\\
		\hline
		5&16&$n\le 50$\\
		\hline
		6&69&\\
		\hline
	\end{tabular}	
\end{center}
\clearpage



\begin{center}
	\large{Medium}
\end{center}
\paragraph{题目背景}
\begin{center}
	\color[rgb]{0.5,0.5,0.5}\emph{——zzy喜欢}
\end{center}
\paragraph{题目描述}
\paragraph{}zzy在他自己身上发现了一种『不会搞颓』的基因,这当且仅当同时存在两个才会表现,这里用\underline{Aa}表示这两种基因型,即\underline{aa}表现出『不会搞颓』
\paragraph{}zzy为了研究,找到了一个家族图谱,它有一个参数$n$,具体为(可以参考样例解释):
\begin{itemize}
	\item 共$2n+2$代,编号为$[0,2n+1]$
	\item 当$i\in[0,n]$时,第$i$代有$2^{i+1}$人,否则有$2^{2n+1-i}$人
	\item 若第$i$代有$2^j$人,则依次编号$[0,2^j)$,$i$代$j$号标记为$(i,j)$
	\item 当$i\in[0,n)$时,对于$k\in[0,2^{j-1})$,有$(i,2k)$与$(i,2k+1)$生出$(i+1,4k),(i+1,4k+1),(i+1,4k+2),(i+1,4k+3)$
	\item 否则,$(i,2k)$与$(i,2k+1)$生出$(i+1,k)$
\end{itemize}
\paragraph{}实验表明:$(0,0),(0,1)$基因型为\underline{Aa},且观察到有$m$个互相没有祖先关系的位于$0\sim n+1$代的人表现为『不会搞颓』,\textbf{但不知道其他人是否有此表现}
\paragraph{}zzy想知道某$q$个位于$n+2\sim 2n+1$代的人各个基因型的概率,他一眼就秒了但不想码代码,于是他命令你帮他解决
\paragraph{输入格式}
\paragraph{}从\emph{medium.in}中读入数据
\begin{itemize}
	\item 第一行三个整数$n,m,q$
	\item 第二行$2m$个整数,表示确定为\underline{aa}的人编号,以$(a,b)$的形式给出
	\item 第三行$2q$个整数,表示希望你计算的人编号,以$(c,d)$的形式给出
\end{itemize}
\paragraph{输出格式}
\paragraph{}输出到\emph{medium.out}中
\begin{itemize}
	\item $q$行每行三个整数,表示\underline{AA,Aa,aa}的概率,对$998244353$取模
\end{itemize}
\paragraph{样例1输入}
\begin{lstlisting}
2 1 1
(2,1)
(4,0)
\end{lstlisting}
\paragraph{样例1输出}
\begin{lstlisting}
720954255 55458020 221832079
\end{lstlisting}
\paragraph{样例1解释}
\paragraph{}概率为$\dfrac{1}{18},\dfrac{7}{18},\dfrac{5}{9}$
\clearpage
\paragraph{样例2}
\paragraph{}见选手文件夹下\emph{/2.in}与\emph{/2.out}
\paragraph{样例3}
\paragraph{}见选手文件夹下\emph{/3.in}与\emph{/3.out}
\paragraph{样例4}
\paragraph{}见选手文件夹下\emph{/4.in}与\emph{/4.out}
\paragraph{样例5}
\paragraph{}见选手文件夹下\emph{/5.in}与\emph{/5.out}
\paragraph{子任务}
\paragraph{}$n\le 60,m\le 100,q\le 100$,下面$a,b,c,d$意义见输入格式,保证$a,b,c,d$合法
\begin{center}
	\begin{tabular}{|c|c|c|c|}
		\hline
		Subtask编号&分值&性质\\
		\hline
		1&1&$n=1$\\
		\hline
		2&2&$n=2$\\
		\hline
		3&3&$m=1$\\
		\hline
		4&3&$a\le 1$\\
		\hline
		5&3&$a\le 2$\\
		\hline
		6&6&$m\le 5$\\
		\hline
		7&6&$m\le 10$\\
		\hline
		8&6&$m\le 20$\\
		\hline
		9&6&$n\le 20$\\
		\hline
		10&64&\\
		\hline
	\end{tabular}	
\end{center}
\clearpage



\begin{center}
	\large{()}
\end{center}
\paragraph{题目描述}
\paragraph{}给定一个长度为$m$数组$a$和一棵$n$个点的树,树上每个点$i$有点权$b_i$,满足$n\leq m$,特别的,定义$c_i$为树上以$i$的子树的最大值
\paragraph{}若一个数$k(1\leq k\leq m-n+1)$合法,当且仅当有一个长度为$n$的排列$p$,有$a_{k+i-1}+c_{p_i}>=h(1\leq i\leq n)$,求合法的$k$的个数
\paragraph{}因为好像太简单,所以有$q$个树上权值修改,每次修改后需要输出合法$k$的个数
\paragraph{输入格式}
\paragraph{}从\emph{.in}中读入数据
\begin{itemize}
	\item 
	\item 
\end{itemize}
\paragraph{输出格式}
\paragraph{}输出到\emph{.out}中
\begin{itemize}
	\item 
\end{itemize}
\clearpage
\paragraph{样例1输入}
\begin{lstlisting}
\end{lstlisting}
\paragraph{样例1输出}
\begin{lstlisting}
\end{lstlisting}
\paragraph{样例1解释}
\paragraph{}
\paragraph{样例2}
\paragraph{}见选手文件夹下\emph{/2.in}与\emph{/2.out}
\paragraph{样例3}
\paragraph{}见选手文件夹下\emph{/3.in}与\emph{/3.out}
\paragraph{样例4}
\paragraph{}见选手文件夹下\emph{/4.in}与\emph{/4.out}
\paragraph{样例5}
\paragraph{}见选手文件夹下\emph{/5.in}与\emph{/5.out}
\clearpage
\paragraph{子任务}
\paragraph{}
\end{document}