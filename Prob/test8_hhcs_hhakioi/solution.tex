\documentclass[UTF8]{ctexart}
\usepackage{graphicx}
\usepackage{booktabs}
\usepackage{listings}
\usepackage{multirow}
\usepackage{mathtools}
\pagestyle{plain}
\begin{document}
\title{solution}
\author{}
\date{\today}
\maketitle

\clearpage

\begin{center}
	\large{Easiest}
\end{center}
\paragraph{}本次考试签到题,改编(简化)自AGC036F
\paragraph{}考虑$l_i=0$为小学奥数题,就不说明过程,结论是(满足$r_i\le r_{i+1}$)
$$
\prod_{i} r_i-i+1
$$
\paragraph{}于是考虑只有$l_{n-1}\not=0$的情况,发现容斥即可
\paragraph{}推广一下,可以$O(2^nn\log n)$容斥解决,即$15~\rm pts$部分分
\paragraph{}考虑题目里的条件,即不存在$i<j$使得$l_i>0,[l_i,r_i]\subset [0,l_j-1]$
\paragraph{}所以可以根据$\begin{cases}r_i&l_i=0\\l_i-1&l_i\not=0\end{cases}$排序
\paragraph{}然后枚举一共选在$[0,l_i)$的位置数,从前到后DP,$f_{i,j}$表示前$i$个选在$[0,l_p)$的位置数为$p$,发现枚举是否在$[0,l_{i+1})$中后就可以直接转移到第$i+1$行了,转移系数参考$l_i=0$的情况,容斥系数为$(-1)^k$

\clearpage

\begin{center}
	\large{Medium}
\end{center}

\paragraph{}一个显然的思路:先找到第$n+1$行的各个基因型概率,然后处理询问
\paragraph{}定义$p(u)$为$u$产生\underline{A}的配子的概率,概率用三元组$P(u)=(AA,Aa,aa)$表示,定义$U_i/V_i$表示第$i$个确定\underline{aa}的人/需要询问的人,先考虑以下性质:

\begin{itemize}
	\item 若$x,y$的$P$相同,则对于他们后代$z$的有为$P(z)=(p^2(x),1-p^2(x)-(1-p(x))^2,(1-p(x))^2)$\\
	      证明: 易证
	\item 若$x,y$的$P$相同,则他们后代$z,w$的后代$u$的$P$与$z,w$相同\\
	      证明: 容易发现$p(x)=p(y)=p(z)=p(w)=p(u)$
\end{itemize}

\paragraph{}若$m=1$,我们可以用上面两个性质算出$U_1$的每个祖先的$P$

\paragraph{}具体地,我们可以算出整体的概率,然后枚举祖先的基因型,用相同方法算概率

\paragraph{}对于每个$U_1$的祖先$u$(包括$U_1$),及其配偶$v$,我们可以得到$P(u),P(v)$

\paragraph{}然后我们可以把第$n+1$层划分为$O(n)$段,每段的$P$相同

\paragraph{}对于询问可以枚举父母的$p$值,这可以直接用第$n+1$代的$P$计算,于是你获得了$3$分的好成绩

\paragraph{}若$m\not=1$,后半部分照旧

\paragraph{}考虑$0\sim n+1$层中,若某一对不存在在$U$里的,那么他们$P$相同,可以认为是一个人

\paragraph{}考虑到不存在$A_i,A_j$使得它们为祖先关系,那么他们及祖先组成一棵树,可以处理出他们的虚树,最多$2m-1$个点

\paragraph{}可以枚举另外$m-1$个点性状,可以$O(3^m\mathrm{poly}(n))$,至此你获得了$15$分

\paragraph{}发现可以DP,于是复杂度变为$O(\mathrm{poly}(m)~\mathrm{poly}(n))$,可以通过此题

\clearpage

\begin{center}
	\large{Hardest}
\end{center}
\end{document}