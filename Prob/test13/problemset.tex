\documentclass[UTF8]{ctexart}
\usepackage{graphicx}
\usepackage{booktabs}
\usepackage{listings}
\usepackage{multirow}
\usepackage{mathtools}
\pagestyle{plain}
\begin{document}
\title{NOI p 普及\textbf{模拟}赛}
\author{}
\date{\today}
\maketitle
\begin{table}[!htbp]
	\centering
	\begin{tabular}{|c|c|c|c|}
		\hline
		题目名称&qiu&sky&dog\\
		\hline
		题目类型&传统型&传统型&传统型\\
		\hline
		目录&qiu&sky&dog\\
		\hline
		可执行文件名&qiu&sky&dog\\
		\hline
		输入文件名&qiu.in&sky.in&dog.in\\
		\hline
		输出文件名&qiu.out&sky.out&dog.out\\
		\hline
		每个测试点时限&1.0s&2.0s&2.0s\\
		\hline
		内存限制&2GB&256MB&512MB\\
		\hline
	\end{tabular}
\end{table}
提交源程序文件名
\begin{table}[!htbp]
	\centering
	\begin{tabular}{|ccc|c|c|c|}
		\hline
		对于&C++&语言&qiu.cpp&sky.cpp&dog.cpp\\
		\hline
		对于&C&语言&qiu.c&sky.c&dog.c\\
		\hline
	\end{tabular}
\end{table}
注意事项:
\begin{enumerate}
	\item 文件名(包括程序名和输入输出文件名)必须使用英文小写。
	\item 结果比较方式为忽略行末空格、文末回车后的全文比较。
	\item C/C++ 中函数 main() 的返回值类型必须是 int,值为 0。
	\item 编译选项为-O2 -std=c++11。
	\item 如果对题目有疑问(如样例出锅),可以找出题人。
	\item \textbf{考试时间8:00至13:00}
\end{enumerate}
\clearpage


\begin{center}
	\large{(qiu)球}
\end{center}
\paragraph{题目描述}
\paragraph{}小P在二维平面的原点$(0,0)$,他现在朝着$y$轴正方向。
\paragraph{}他会以如下方式放$10^{10^{100}}$个球,第$i$个的重量为$i$:
\paragraph{}在$i$步,他会放下第$i$个球;若他的右方第一个整点没有放球,那么向右转;向前走一单位长度。
\paragraph{}前25步为:
\begin{lstlisting}
21 22 23 24 25
20 7  8  9  10
19 6  1  2  11
18 5  4  3  12
17 16 15 14 13
\end{lstlisting}
\paragraph{}小P回询问$q$次,所有横坐标在$[x_1,x_2]$中,纵坐标在$[y_1,y_2]$中的整点,上面球的重量的和,答案可能很大,对$2^{63}$取模。
\paragraph{输入格式}
\paragraph{}从\emph{qiu.in}中读入数据
\begin{itemize}
	\item 第一行一个整数$q$。
	\item 接下来$q$行,每行四个整数$x_1,x_2,y_1,y_2$。
\end{itemize}
\paragraph{输出格式}
\paragraph{}输出到\emph{qiu.out}中
\begin{itemize}
	\item 一行一个整数,你的答案。
\end{itemize}
\clearpage
\paragraph{样例输入}
\begin{lstlisting}
1
0 0 0 1
\end{lstlisting}
\paragraph{样例输出}
\begin{lstlisting}
9
\end{lstlisting}
\paragraph{样例解释}
\paragraph{} $1+8=9$。
\paragraph{数据规模}
\paragraph{} 令$W$为满足以下条件的最小正整数:$0\le x_1,x_2,y_1,y_2\le W$。
\paragraph{} 有:$W\le 10^{18},x_1\le x_2,y_1\le y_2,q\le 10^6$。
\begin{itemize}
	\item Subtask 1(1'): $W\le 10^4$。
	\item Subtask 2(2'): $W\le 10^7$。
	\item Subtask 3(3'): $W\le 10^9$。
	\item Subtask 4(4'): $W\le 10^{12}$。
	\item Subtask 5(5'): $q=1$。
	\item Subtask 6(85'): $W\le 10^{18}$。
\end{itemize}
\clearpage


\begin{center}
	\large{(sky)星空}
\end{center}
\paragraph{题目背景}
\paragraph{}2021元旦,小P注视着星空,眼睑通红
\paragraph{}苦中带涩不是常态吗,我们得用笑容面对苦涩
\paragraph{}使劲揉揉眼睛,我将不负过去,不畏将来
\paragraph{题目描述}
\paragraph{}星空中充斥着星辰,小P想将其一一配对,一个星辰用两个参数描述$a_i,b_i$分别表示其大小和闪烁程度。
\paragraph{}两个星辰$i,j$可以配对,当且仅当$a_i\leq a_j$并且获得$b_j-b_i$的整体闪烁程度(因为小P眼睑通红看不清楚,所以会相互抵消)
\paragraph{}一个星辰配对后无法再次配对,小P想知道对于$k\in[1,\lfloor \frac{n}{2}\rfloor]$求出至多配对$k$的最大闪烁程度和
\paragraph{输入格式}
\paragraph{}从\emph{sky.in}中读入数据
\begin{itemize}
	\item 一行一个整数$n$表示$n$颗星辰
	\item 接下来$n$行,每行2个整数分别表示$a_i,b_i$
\end{itemize}
\paragraph{输出格式}
\paragraph{}输出到\emph{sky.out}中
\begin{itemize}
	\item 共$\lfloor\frac{n}{2}\rfloor$行,第$i$行表示$k=i$的答案
\end{itemize}
\clearpage
\paragraph{样例输入}
\begin{lstlisting}
6
1 1
4 5
1 4
1 9
5 9
8 10
\end{lstlisting}
\paragraph{样例输出}
\begin{lstlisting}
9
14
18
\end{lstlisting}
\paragraph{数据规模}
\paragraph{} 对于所有数据,满足$n\leq 10^5,a_i\leq 10^9,b_i\leq 10^9$
\begin{itemize}
	\item Subtask 1(13'): $n\leq 100$。
	\item Subtask 2(9'): $n\leq 1000$。
	\item Subtask 3(7'): 所有$a_i$均相等。
	\item Subtask 4(18'): $b_i\leq 1000$。
	\item Subtask 5(13'): $n\le 5000$。
	\item Subtask 6(40')
\end{itemize}
\clearpage

\begin{center}
	\large{(dog)嫖怪}
\end{center}
\paragraph{题目背景}
\paragraph{}社会生产力的发展,使得人们的消费能力与需求与日俱增。
\paragraph{}正因如此,小P搬运食品的速度已经远远无法跟上机房嫖怪们的需求了
\paragraph{}若再不想些对策,小P那点可怜的劳动价值也要被血腥的嫖怪资本家们榨取殆尽了……
\paragraph{题目描述}
\paragraph{}小P用食品贮藏点的方式储存自己的食品,且一开始没有任何贮藏点。接下来会有$q$个操作。
\paragraph{}$type=2$:对抗操作。
\paragraph{}为了应对可能发生的事件,小P会模拟自己与嫖怪们的对抗过程。
\paragraph{}具体地,对抗从$0s$时刻开始,在时间段$(ts,ts+1s],t\in N$间,小P会在$ts+0.5s$时选择一个贮藏点,向其中投放$1$单位食品。嫖怪们则会在$ts+1s$时选择一个贮藏点,至多取出$2$单位的食品。
\paragraph{}如果一个食品贮藏点贮藏的食品单位数$\leq 0$,嫖怪们就会认为它不再具有利用价值并摧毁它。若被摧毁的食品贮藏点达到了$K$个,小P就不得不修建新储藏点了。
\paragraph{}但是小P很懒惰,因此他希望被摧毁点数达到$K$个的时刻尽量大。小P认为嫖怪们也很懒惰,因此小P认为他们希望这个时刻尽量小。
\paragraph{}对于这个操作,你需要输出对抗结束的时刻$T$,这显然是个整数。
\paragraph{}这个操作不具有后效性。
\paragraph{}$type=1$:修改操作。
\paragraph{}小P会将储存食品数为$X$的贮藏点的数目改为$Y$。
\paragraph{}这个操作具有后效性。
\paragraph{输入格式}
\paragraph{}从\emph{dog.in}中读入数据
\begin{itemize}
	\item 一行一个整数$q$表示$q$次询问
	\item 接下来$q$行,每行第一个整数$type$,表示操作类型。
	\item 若type=1,则接下来两个正整数$X$,$Y$,表示小P的修改,意义同题目描述。
	\item 若type=2,则接下来一个正整数$K$,表示小P的模拟对抗,意义同题目描述。
\end{itemize}
\paragraph{输出格式}
\paragraph{}输出到\emph{dog.out}中
\begin{itemize}
	\item 对于每个type=2的操作,输出一个整数$T$,表示小P模拟对抗结束的时刻。
	\item 注意:每个type=2的操作是小P的模拟对抗,不具有后效性。
\end{itemize}
\clearpage
\paragraph{样例1输入}
\begin{lstlisting}
8
1 1 1
2 1
1 2 3
2 3
1 2 2
2 3
1 3 3
2 4
\end{lstlisting}
\paragraph{样例1输出}
\begin{lstlisting}
1
4
5
6
\end{lstlisting}
\paragraph{样例1解释}
\paragraph{}下面用$(X,Y)$表示一个据点。其中$X$为编号,$Y$为初始储存食品数,操作序列中只有编号,且按时间排序。
\paragraph{}对于第二个$type=2$的操作。现有据点为$(1,1),(2,2),(3,2),(4,2)$,
\paragraph{}那么小P的操作序列为$4,3,3,1$,嫖怪的操作序列则为$2,3,3,1$。这是合乎题意的,且对双方而言都最优。
\paragraph{样例2}
\paragraph{}见选手文件夹下\emph{/2.in}与\emph{/2.out}。他满足子任务$2$的限制。
\paragraph{样例3}
\paragraph{}见选手文件夹下\emph{/3.in}与\emph{/3.out}。他满足子任务$3$的限制。
\paragraph{样例4}
\paragraph{}见选手文件夹下\emph{/4.in}与\emph{/4.out}。他满足子任务$4$的限制。
\paragraph{数据规模}
\paragraph{} 对于所有数据,满足$q\leq 10^{6},type\in{1,2}, X,Y\leq 10^{6}, K\leq$ 当前贮藏点的总数
\begin{itemize}
	\item Subtask 1(10'): $q\leq 5, X,Y\leq 3$。
	\item Subtask 2(30'): 所有$X$均相等。
	\item Subtask 3(10'): $q\leq 1000, X\leq 1000, Y\leq 10^{6}$。
	\item Subtask 4(50'): 无特殊限制。
\end{itemize}
\end{document}