\begin{frame}{作为今天的开胃小水题}
	今天来讲一个经典题\pause
	的加强版。

	\pause
	可能有人见过了,不过放在联赛题目的宣讲小清新开场还不错。
\end{frame}
\section{某题}
\begin{frame}{Statement}
	给定一棵大小为$n$的有标号无根树和一参数$k$。
	
	求有几棵树的边集的与原树边集的交集大小不小于$k$。

	经典范围:$n\le 100$

	加强范围:$n\le 5000$
\end{frame}
\section{naive solution}
\begin{frame}{$n\le 100$}
	考虑矩阵树定理。

	对大小为$n$的完全图运用矩阵树定理:把原树中出现的边权设为$x$,其他的边设为$1$。\pause

	容易发现,求出的行列式是一个关于$x$的多项式$f(x)$,求出$\sum_{i\geq k}[x^i]f(x)$即可。

	暴力求$f(x)$也不难,枚举点值然后插值就行了。\pause

	$\mathcal O(n^4)$
\end{frame}
\section{better solution}
\begin{frame}{$n \le 5000$}
	矩阵树定理好像没啥前途。\pause

	如果我们现在已经钦定了一些边一定相同(即一定出现在交集中),怎么统计方案呢?

	现在我们面临的局面是:有$n$个点形成了$k$个连通块,第$i$个连通块的大小为$s_i$,求把它们连成一棵树的方案数。\pause

	朋友,你看过\href{https://oi-wiki.org/graph/prufer/}{oi-wiki上介绍prufer的页面吗}?

	\begin{center}
		\includegraphics[width=10cm]{prufer-query.png}

		\includegraphics[width=10cm]{prufer-answer.png}
	\end{center}
	\pause

	具体证明可以考虑枚举prufer序列,oi-wiki上讲解得非常详细,不再赘述。
\end{frame}
\begin{frame}{$n \le 5000$}
	所以我们现在面临的问题变成了:求所有钦定$m$条边相同时,$\prod a_i$的和。
	\pause
	
	朋友,你听说过WC[2019]数树吗?

	$a_i$的组合意义也就是“在$i$号连通块里面选出一个点。”
	\pause

	好做了,设$f_{u,l,0/1}$表示“$u$的子树内钦定$l$条边相同,当前连通块选不选点”的方案数,dp即可。

	最后在套一个二项式反演就行啦。\pause

	$\mathcal O(n^2)$
\end{frame}