\documentclass[9pt]{beamer}
\usepackage{xeCJK}
\usepackage[space,space,hyperref]{ctex}
\setsansfont{Ubuntu Mono}
\setCJKmonofont{AR PL UKai CN}
\usetheme{metropolis}
\title{博弈论}
\date{\today}
\author{ljfcnyali}
\begin{document}
  \maketitle

  \section{纳什均衡}

  \begin{frame}{纳什均衡}
    \onslide<1-> 考虑一组双人博弈游戏,给定矩阵 $A_{i,j},B_{i,j}$ 分别表示 Alice 选择 $i$ 而 Bob 选择 $j$ 时的收益,同时保证总收益为0。
  
    \onslide<2-> 而纳什均衡点指在 Alice 决定策略的情况下, Bob 不可以通过改变策略增加自己的期望收益。

    \onslide<3-> 即假设一组向量 $x,y$ 分别表示 Alice 和 Bob 的决策概率,那么对于任意 $u,v$ 有 $u^TAy\leq x^TAy,x^TBv\leq x^TBy$。
  \end{frame}

  \begin{frame}{纳什均衡}
    \onslide<1-> 给出一个例子:Alice 与 Bob 分别选取一个硬币的正反面,我们规定:Alice正Bob正收益$a$, Alice反Bob反收益$b$,Alice反Bob正收益$-c$,否则收益$-d$,求解纳什均衡策略。
  
    \onslide<2-> 考虑设 Alice 有 $x$ 概率选择正, $1-x$ 概率选择反,由定义有 Bob 无论怎么选都不会对 Alice 期望收益造成影响,故:
    
    \onslide<3-> $$ xa-(1-x)c=-xd+(1-x)b $$

    \onslide<4-> 求解出 $x$ 即可。

    \onslide<5-> 稍微解释一下,纳什均衡不是说要让我的收益最大化,而是说期望意义下收益最大。即双方的决策概率都是互相已知的。
  \end{frame}

  \begin{frame}{纳什均衡}
    \onslide<1-> 通常来说,我们使用线性规划来求解纳什均衡点。

    \onslide<2-> 由前文可知,我们只需要设 $x_i$ 表示 Alice 选择 $i$ 的概率,$ans$表示最优策略下的期望收益,最大化 $ans$,限制如下:
    
    \onslide<3-> 
    $$
    \begin{aligned}
      \sum x_i &\leq 1\\
      \sum_j v_{i,j}x_i &\geq ans \\
      x_i&\geq 0
    \end{aligned}
    $$

    \onslide<4-> 其它情况下特殊考虑即可,同时注意在最终答案下, $\sum_j v_{i,j}x_i=ans$ ,容易证明。

    \onslide<5-> 当然还有很多其他方法求解纳什均衡,但是因为讲课人水平有限,故不再探究。
  \end{frame}
  
  \section{杂题选讲}

  \begin{frame}{来源不明的签到题}
    \par 给定一个字符串 $s$,Alice 和 Bob 分别同时独立选择一个后缀,并且计算两个后缀的最长公共前缀。Alice 希望它尽量大,Bob希望尽量小,询问期望长度。
    \par $|s| \leq 10^5$
  \end{frame}
 
  \begin{frame}{来源不明的签到题}
    \onslide <1-> 很明显,来一手SA后问题转化为了选择的两个后缀 $[l,r]$ 的 $min_{i=l}^r\{height_i\}$ 的期望。

    \onslide <2-> 考虑按照最小值分治,假设当前只处理后缀在区间 $[l,r]$ 内的期望,$val_l$ 表示左区间答案,$val_r$ 表示右区间答案, $Min$ 表示 $height$ 最小值,$x$ 表示 Alice 选择左区间的概率,列出方程有:

    \onslide <3-> $$ x\times val_l + (1-x) \times Min = (1-x) \times val_r + x \times Min$$

    \onslide <4-> 求解出 $x$ 后再算出等式取值,然后就做完了。
  \end{frame}
  
  \begin{frame}{CF1091H}
    \par 有一个 $n$ 行无限列的棋盘,每一行都有三个棋子,从左到右依次为蓝白红,位置为 $b_i,w_i,r_i$ 。现在有 Alice 和 Bob 依次操作,每次 Alice 可以将某一行的蓝或蓝白棋同时右移 $k$ ,而 Bob 可以将某一行的红或白红棋同时左移 $k$。
    \par 需要满足 $k$ 是质数或为两个质数的乘积,且每一行棋子相对顺序不变,求先手是否必胜。
    \par $n \le 10^5, -10^5\leq b_i,w_i,r_i\leq 10^5$

    \onslide<2-> Hint:这题是不平等博弈吗?
  \end{frame}
  
  \begin{frame}{CF1091H}
    \onslide<1-> 首先考虑一行的情况,因为列数无穷,所以只需要考虑棋子的相对顺序。

    \onslide<2-> 而我们发现, Alice 的操作本质上是将两个相对位置中缩小其中一个, Bob 同理,故两人操作本质相同,该题为平等博弈。

    \onslide<3-> 可以抽象为每一行有两堆石子,轮流操作,每次可以将一堆删去 $k$。而求一堆石子的 sg 函数可以通过 $bitset$ 简单的计算,具体来说,维护每个 $sg_i$ 可以由哪些石子转移即可。

    \onslide<4-> 最后各行因为互不干扰,异或判断答案即可。
  \end{frame}

  \section{Thanks}

\end{document}
