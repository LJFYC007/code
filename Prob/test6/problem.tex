 
\documentclass[UTF8]{ctexart}
\usepackage{booktabs}
\usepackage{amsmath}
\usepackage{geometry}
\usepackage{fancyhdr}
\usepackage{listings}
\usepackage{color}

\begin{document}
\title{良心CSP-S模拟赛}
\author{Adscn}
\date{2019年11月??日}
\maketitle

\begin{table}[!htbp]
	\centering
	\begin{tabular}{|c|c|c|c|}
		\hline
		题目名称&简单的推式子练习题&简单的字符串练习题&简单的杜教筛练习题\\
		\hline
        题目类型&传统型&传统型&传统型\\
        \hline
        目录&formula&string&interval\\
        \hline
        可执行文件名&formula&string&interval\\
        \hline
        输入文件名&formula.in&string.in&interval.in\\
        \hline
        输出文件名&formula.out&string.out&interval.out\\
        \hline
        每个测试点时限&3s&1s&7s\\
        \hline
        内存限制&512MB&512MB&512MB\\
        \hline
        测试点/包数目&20&20&10\\
        \hline
        测试点是否等分&是&是&是\\
		\hline
    \end{tabular}
\end{table}

注意事项:

1.该比赛为\sout{良心模拟赛}

2.本考试所有目录,文件名均为\textbf{小写}

3.本考试\sout{不保证}数据正确性

4.所有题目的时间限制在std的1.5倍以上

5.在$Ubuntu\ 18.04$下评测,开启O2优化,启用--std=c++11

6.本场考试送分到位,请AK的同学低调离场,不要声张。

7.Orz $\color{black}\texttt{Q}\color{red}\texttt{AQAutomaton}$

8.请注意64位整数的运算效率问题

9.题目其实很\textbf{简单}

\clearpage

\begin{center}
    \large{简单的推式子练习题}
\end{center}
\paragraph{题目描述}
\paragraph{}
给出一颗树,点有点权,请支持以下操作

$1.$换根并询问\textbf{所有子树}的\textbf{带边权点权和}

$2.$修改一个点的权值

\textbf{带边权点权和}定义:

设$Sum_x$为\textbf{子树}内的\textbf{点权和}

一个子树$x$的带边权点权和为

$dep_x*Sum_x$

形式化地,询问的值是
$$
    \sum\limits_{x\in Tree} dep_x\sum\limits_{v\in subtree\, of\,x}{val_v}
$$

\paragraph{输入格式}
\paragraph{}
第一行两个数字$n,q$表示树的节点个数与操作个数

接下来的 $n-1$ 行,每行包含两个正整数 $u$ 和 $v$ ($1\le u,v\le n$),表示 $u$ 和 $v$ 之间由一条边相连。

第$n+1$行,有$n$个数字$a_1,a_2,a_3\ldots a_n$,表示$n$个点的点权($|a_i|\leq 10^9$)

接下来的 $q$ 行,每行的最开始包含一个整数$opt$

若$opt=1$ 接下来有两个数$x,y$,表示将$x$号点的点权修改成$y$

否则 接下来有一个数$x$,表示询问以$x$为根的带边权点权和(对$10^9+7$取模)

数据保证给出的边能构成一棵树。

\paragraph{输出格式}
\paragraph{}对于每个操作2输出一行答案在$mod$$\,{10^9+7}$意义下的最小正整数
\paragraph{样例1输入}
\begin{lstlisting}
    10 4
    1 2
    1 3
    2 6
    2 4
    4 8
    8 9
    3 5
    3 7
    5 10
    2 1 3 4 2 3 4 2 1 -1
    2 1
    1 8 5
    2 1
    2 6
\end{lstlisting}
\paragraph{样例1输出}
\begin{lstlisting}
    59
    77
    122
\end{lstlisting}
\paragraph{数据范围}
\paragraph{}Subtask1.对于$20\%$的数据,$n,q\leq 2000$
\paragraph{}Subtask2.对于$40\%$的数据,$n,q\leq 50000$
\paragraph{}Subtask3.对于另外$10\%$的数据,保证图的形态为菊花
\paragraph{}Subtask4.对于另外$10\%$的数据,保证图的形态为链
\paragraph{}Subtask5.对于另外$10\%$的数据,数据随机
\paragraph{}对于$100\%$的数据,$n,q\leq 3*10^5$
\paragraph{提示}
\paragraph{}题目其实很\textbf{套路}

\clearpage

\begin{center}
    \large{简单的字符串练习题}
\end{center}
\paragraph{题目描述}
\paragraph{}现在有一个长度为$n$的01串$s$,你需要构造长度为$m$的01串,使得其子串与原串$s$\textbf{至多}有一个位置不同,询问构造长度为$m$的01串的方案数
\paragraph{输入格式}
\paragraph{}首先一行一个数$T$表示数据组数
\paragraph{}对于每组数据第一行两个数$n$,$m$
\paragraph{}接下来一个长度为$n$的01串$s$
\paragraph{输出格式}
\paragraph{}共$T$行,每行一个数表示答案(对998244353取模)
\paragraph{样例1输入}
\begin{lstlisting}
    3
    2 3
    10
    4 6
    1010
    6 9
    010110
\end{lstlisting}
\paragraph{样例1输出}
\begin{lstlisting}
    8
    49
    208
\end{lstlisting}
\paragraph{}
\paragraph{样例解释}
\paragraph{}因为答案过大,所以只对第一个数据进行解释:
\paragraph{}可以构造出来的串$m$有$111,110,101,100,011,010,001,000$(其实都可以$\dots$)
\paragraph{数据范围}
\paragraph{}对于$20\%$的数据,$n,m\leq 10$,$T=1$
\paragraph{}对于$50\%$的数据,$n,m\leq 20$,$T\leq 10$
\paragraph{}对于另外$10\%$的数据,$n=m$且$n,m\leq 50$,$T\leq 100$
\paragraph{}对于$100\%$的数据,$n,m\leq 200$,$T\leq 100$

\paragraph{提示}
\paragraph{}题目并\textbf{不难}
\clearpage

\begin{center}
    \large{简单的杜教筛练习题}
\end{center}
\paragraph{题目描述}
\paragraph{}
给你两个数$l,r$,请你求
$$
\sum\limits_{i=l}^r{\sigma(i)}
$$
其中
$$
\sigma(x)=\sum\limits_{d|x} d
$$
\paragraph{输入格式}
\paragraph{}第一行两个数$l$,$r$
\paragraph{输出格式}
\paragraph{}一个数表示答案(对$10^9+7$取模)
\paragraph{样例1输入}
\begin{lstlisting}
    10 30
\end{lstlisting}
\paragraph{样例1输出}
\begin{lstlisting}
    693
\end{lstlisting}
\paragraph{数据范围}
\paragraph{}对于$20\%$的数据,$l,r\leq 6*10^5$
\paragraph{}对于$40\%$的数据,$l,r\leq 2*10^{12}$
\paragraph{}对于$100\%$的数据,$l,r\leq 10^{18},r-l\leq10^5$
\paragraph{提示}
\paragraph{}这是一道良心\textbf{大水题}
\end{document}