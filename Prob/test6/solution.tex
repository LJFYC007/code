 
\documentclass[UTF8]{ctexart}
\usepackage{booktabs}
\usepackage{amsmath}
\usepackage{geometry}
\usepackage{fancyhdr}
\usepackage{listings}
\usepackage{color}

\begin{document}
\title{良心CSP-S模拟赛}
\author{Adscn}
\date{2019年11月??日}
\maketitle

\begin{table}[!htbp]
	\centering
	\begin{tabular}{|c|c|c|c|}
		\hline
		题目名称&简单的推式子练习题&简单的字符串练习题&简单的杜教筛练习题\\
		\hline
        题目类型&传统型&传统型&传统型\\
        \hline
        目录&formula&string&interval\\
        \hline
        可执行文件名&formula&string&interval\\
        \hline
        输入文件名&formula.in&string.in&interval.in\\
        \hline
        输出文件名&formula.out&string.out&interval.out\\
        \hline
        每个测试点时限&3s&1s&7s\\
        \hline
        内存限制&512MB&512MB&512MB\\
        \hline
        测试点/包数目&20&20&10\\
        \hline
        测试点是否等分&是&是&是\\
		\hline
    \end{tabular}
\end{table}

注意事项:

1.该比赛为\sout{良心模拟赛}

2.本考试所有目录,文件名均为\textbf{小写}

3.本考试\sout{不保证}数据正确性

4.所有题目的时间限制在std的1.5倍以上

5.在$Ubuntu\ 18.04$下评测,开启O2优化,启用--std=c++11

6.本场考试送分到位,请AK的同学低调离场,不要声张。

7.Orz $\color{black}\texttt{Q}\color{red}\texttt{AQAutomaton}$

8.请注意64位整数的运算效率问题

9.题目其实很\textbf{简单}

\clearpage

\begin{center}
    \large{简单的推式子练习题}
\end{center}
\paragraph{题目描述}
\paragraph{}
给出一颗树,点有点权,请支持以下操作

$1.$换根并询问\textbf{所有子树}的\textbf{带边权点权和}

$2.$修改一个点的权值

\textbf{带边权点权和}定义:

设$Sum_x$为\textbf{子树}内的\textbf{点权和}

一个子树$x$的带边权点权和为

$dep_x*Sum_x$

形式化地,询问的值是
$$
    \sum\limits_{x\in Tree} dep_x\sum\limits_{v\in subtree\, of\,x}{val_v}
$$

\paragraph{题解}
\paragraph{}

单独考虑每个点$x$的贡献,发现它会被计算$\frac {dep_x*(dep_x+1)}2$次

那题目让我们求的,其实就是$\sum\limits_{v\in tree} \frac {dis(u,v)*dis(u,v)+dis(u,v)}2 val_v$(u为给定根)

先去掉那个$\frac 12$,发现就是求$\sum dis(u,v) val_v$与$\sum dis(u,v)^2 val_v$

对于前一部分,考虑对于每个子树$x$维护$\sum\limits_{v\in x} dis(x,v) val_x$,从$u$出发,在向上跳的过程中维护答案即可,$dis(x,v)^2$同理。

这样就可以拿到随机树的分,但是链会被卡爆。

考虑优化。

我们使用动态点分治,对于树建出一颗点分树出来,前面的不带平方的做法就是ZJOI的幻想乡战略游戏,这里只主要讲解带平方的部分如何计算。

以下如果没有特殊说明都指点分树。

设$dis1_x$为点$x$其子树中的点到达$x$的无平方代价和,$dis2_x$为$x$的子树到达$par_x$的无平方代价和。

那对于前一部分我们跳跃的时候就是直接容斥就可以了。

我们设$dis3_x$为到达$x$的平方代价和,$dis4_x$为到达$par_x$的平方代价和,$sum_x$为点分树内的点权和。

这一次的在点分树上跳跃后,当前点到$u$的距离为$dis$

答案显然就先要加上$dis3_x$,减去$dis4_x$

我们参考无平方的容斥思想,是先把到达$u$点的代价转化为到达在点分树上当前点$x$的父亲$y$代价,然后再用维护的值算出答案。

就是把跳跃的新增的点到$u$的代价转化为到$y$的代价。

考虑现在从$x$跳跃到$y$

原来计算的是$x$部分的点到达$u$的代价。

考虑$y$新增的点到达$u$的代价转化为到达$y$的代价。

毫无疑问是,

$\sum\limits_{v\in y \and v\notin x}dis(u,v)^2val_v-\sum\limits_{v\in y\and v\notin x}dis(v,y)^2val_v$

$\sum\limits_{v\in y \and v\notin x}(dis(u,v)^2-dis(v,y)^2)val_v$

考虑我们已知的信息。

$dis1_x=\sum\limits_{v\in x}dis(v,x)val_v$

$dis2_x=\sum\limits_{v\in x}dis(v,x)^2val_v$

$dis3_x=\sum\limits_{v\in x}dis(v,par_x)val_v$

$dis4_x=\sum\limits_{v\in x}dis(v,par_x)^2val_v$

$sum_x=\sum\limits_{v\in x}val_x$

现有$dis(u,y)^2(sum_{par_x}-sum_x)=\sum\limits_{v\in y \and v\notin x}dis(u,y)^2 val_v$

剩下的部分就是

$\sum(dis(v,y)^2-dis(u,y)^2)val_v=\sum((dis(v,y)+dis(y,u))^2-dis(u,y)^2)val_v$

设$dis=dis(u,y)$

最后就是

$2dis(\sum\limits_{v\in y\and v\notin x}dis(v,y)val_v)+dis^2(\sum\limits_{v \in y\and v\notin x}val_v)$

就可以直接做了。

\clearpage

\begin{center}
    \large{简单的字符串练习题}
\end{center}
\paragraph{题目描述}
\paragraph{}现在有一个长度为$n$的01串$s$,你需要构造长度为$m$的01串,使得其子串与原串$s$\textbf{至多}有一个位置不同,询问构造长度为$m$的01串的方案数
\paragraph{题解}
\paragraph{50pts}
\paragraph{}$2^n$暴力
\paragraph{100pts}
\paragraph{}
首先将符合条件的长度为 n 的串构建一个有 n + 1 个串的 AC 自动机

考虑 DP,令 dp[i][j] 表示构造到第 i 位,AC 自动机匹配到了第 j 位不符合答案的方案数,
可以想到转移方程 $dp[i][j] =\sum dp[i − 1][k]$(k 表示 AC 自动机上 k 可以转移到 j,其中如果AC 自动机上 k 为某一个串的末尾,则不添加 dp[i − 1][k])

可以想到这个方法的正确性,因为在每一次 DP 过程中可以保证不出现可以与 AC 自动机

完全匹配的串,其实在 DP 的过程中我们相当于在模拟串的匹配

所以最后的答案就是 $2^m −\sum dp[m][i]$

时间复杂度$O(n^3)$
\clearpage
\begin{center}
    \large{简单的杜教筛练习题}
\end{center}
\paragraph{题目描述}
\paragraph{}
给你两个数$l,r$,请你求
$$
\sum\limits_{i=l}^r{\sigma(i)}
$$
其中
$$
\sigma(x)=\sum\limits_{d|x} d
$$
\paragraph{题解}
\paragraph{20pts}
\paragraph{}暴力即可
\paragraph{50pts}
\paragraph{}依标题模拟即可。
\paragraph{100pts}
\paragraph{}
注意到$r-l$很小,于是我们可以先筛出$1e6$以内的质数,筛去贡献,

然后考虑每个数,如果它没有$>1e6$的质因子,就跳过即可。

否则,分两种情况讨论,设剩下的数为$x$

$1.x=p$(p是质数),直接算上贡献即可,用$Miller\_Rabin$判断。

$2.x=pq$(p,q是质数),暴力$Pollard\_Rho$即可。

$3.x=p^2$(p是质数),直接用sqrt判断即可。
\end{document}