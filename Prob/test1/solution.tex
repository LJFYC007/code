\documentclass[UTF8]{ctexart}
\usepackage{booktabs}
\usepackage{amsmath}
\usepackage{geometry}
\usepackage{fancyhdr}
\usepackage{listings}

\begin{document}
\paragraph{$20\mathrm{pts}$}
\paragraph{}给常数不优秀的同学
\paragraph{$50\mathrm{pts}$}
\paragraph{}暴力枚举每一位,然后判断是否符合条件,时间复杂度$O(2^n)$
\paragraph{另外$10\mathrm{pts}$}
\paragraph{}$n=m$的情况很明显答案是$n+1$
\paragraph{$100\mathrm{pts}$}
\paragraph{}首先将符合条件的长度为$n$的串构建一个有$n+1$个串的AC自动机
\paragraph{}考虑DP,令$dp[i][j]$表示构造到第$i$位,AC自动机匹配到了第$j$位不符合答案的方案数,可以想到转移方程$dp[i][j]=\sum{dp[i-1][k]}$($k$表示AC自动机上$k$可以转移到$j$,其中如果AC自动机上$k$为某一个串的末尾,则不添加$dp[i-1][k]$)
\paragraph{}可以想到这个方法的正确性,因为在每一次DP过程中可以保证不出现可以与AC自动机完全匹配的串,其实在DP的过程中我们相当于在模拟串的匹配
\paragraph{}所以最后的答案就是$2^m-\sum dp[m][i]$(其中$i$为AC自动机的每个节点),时间复杂度为$O(n^3)$
\end{document}