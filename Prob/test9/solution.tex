\documentclass[UTF8]{ctexart}
\usepackage{booktabs}
\usepackage{amsmath}
\usepackage{geometry}
\usepackage{fancyhdr}
\usepackage{listings}

\begin{document}
\begin{center}Easiest\end{center}
\paragraph{}考虑暴力计算阶乘,出现$0$退出
\begin{center}Tree\end{center}
\paragraph{}观察对于一个子树的根$x$,可以将其儿子根据形态分开来做,对于第$i$种形态,设$f_i$表示该种形态的答案,$cnt_i$表示该种形态的子树个数,则有$ans=\sum_{k=1}^{cnt_i}\binom{cnt_i-1}{k-1}\times \binom{f_i}{k}$。
\paragraph{}对于不同的形态可以用乘法原理合并即可,再乘上根的方案数$m$,该做法复杂度实际为$O(n)$
\paragraph{}课后作业:想一想无根树怎么做?
\begin{center}Horseless\end{center}
\paragraph{}设$f(x)$表示在$x$时刻可以获得的最多码,另外观察题意,我们购买的码力自动机一定是价格递增的,所以考虑将所有码力自动机按价格递增排序,再依次考虑是否购买
\paragraph{}假设$f(x)$是一个函数,那么我们肯定是要维护一个上凸壳,而购买一个码力自动机会对未来造成的贡献一定是一条直线,那么只需要用李超线段树简单的维护一下即可,复杂度为$O(nlog^2_n)$
\paragraph{}但是冷静分析一下该题的特殊性,一定有斜率单调递增,所以可以拿双指针做到维护凸壳是$O(n)$的复杂度,加上排序复杂度为$O(nlog_n)$
\end{document}