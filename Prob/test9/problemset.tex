\documentclass[UTF8]{ctexart}
\usepackage{graphicx}
\usepackage{booktabs}
\usepackage{listings}
\usepackage{multirow}
\usepackage{color}
\usepackage{mathtools}
\pagestyle{plain}
\begin{document}
\title{NOIP提高组Day2}
\author{}
\date{\today}
\maketitle
\begin{table}[!htbp]
	\centering
	\begin{tabular}{|c|c|c|c|}
		\hline
		题目名称&Easiest&Tree&Horseless\\
		\hline
		题目类型&传统型&传统型&传统型\\
		\hline
		目录&easiest&tree&horseless\\
		\hline
		可执行文件名&easiest&tree&horseless\\
		\hline
		输入文件名&easiest.in&tree.in&horseless.in\\
		\hline
		输出文件名&easiest.out&tree.out&horseless.out\\
		\hline
		每个测试点时限&1.0s&1.0s&6.0s\\
		\hline
		内存限制&512MB&512MB&512MB\\
		\hline
		测试点/包数目&1&5&5\\
		\hline
		测试点是否等分&是&是&是\\
		\hline
	\end{tabular}
\end{table}
提交源程序文件名
\begin{table}[!htbp]
	\centering
	\begin{tabular}{|ccc|c|c|c|}
		\hline
		对于&C++&语言&easiest.cpp&tree.cpp&horseless.cpp\\
		\hline
		对于&C&语言&easiest.c&tree.c&horseless.c\\
		\hline
	\end{tabular}
\end{table}
注意事项:
\begin{enumerate}
	\item 文件名(包括程序名和输入输出文件名)必须使用英文小写。
	\item 结果比较方式为忽略行末空格、文末回车后的全文比较。
	\item C/C++ 中函数 main() 的返回值类型必须是 int,值为 0。
	\item 编译选项为-O2 -std=c++11
	\item 如果对题目有疑问(如样例出锅),可以找出题人
\end{enumerate}
\clearpage


\begin{center}
	\large{Easiest}
\end{center}
\paragraph{题目描述}
\paragraph{}给定一个整数$n$,求$n$的阶乘,对$99$取模
\paragraph{输入格式}
\paragraph{}从\emph{easiest.in}中读入数据
\begin{itemize}
\item 一行一个整数$n$
\end{itemize}
\paragraph{输出格式}
\paragraph{}输出到\emph{easiest.out}中
\begin{itemize}
	\item 一行一个整数表示答案
\end{itemize}
\paragraph{样例输入}
\begin{lstlisting}
2
\end{lstlisting}
\paragraph{样例输出}
\begin{lstlisting}
2
\end{lstlisting}
\paragraph{子任务}
\paragraph{}对于$50\%$的数据,满足$n\le 10$
\paragraph{}对于$100\%$的数据,满足$n\le 10^{18}$

\clearpage

\begin{center}
	\large{Tree}
\end{center}
\paragraph{题目描述}
\paragraph{}给定一棵$n$个点的以$1$为根的有根树,现在有$m$种颜色,你需要对每个节点染色。求本质不同的染色数,对$998244353$取模。
\paragraph{}本质相同定义为:忽略节点编号后(根不变),两棵树同构(颜色+形态)
\paragraph{输入格式}
\paragraph{}从\emph{tree.in}中读入数据
\begin{itemize}
\item 第一行两个整数$n,m$
\item 接下来$n-1$行每行两个整数$u,v$表示$u$与$v$有一条边
\end{itemize}
\paragraph{输出格式}
\paragraph{}输出到\emph{tree.out}中
\begin{itemize}
	\item 一行一个整数表示染色方案数
\end{itemize}
\paragraph{样例输入}
\begin{lstlisting}
5 4
1 2
1 3
2 4
2 5
\end{lstlisting}
\paragraph{样例输出}
\begin{lstlisting}
640
\end{lstlisting}
\paragraph{子任务}
\paragraph{}对于$100\%$的数据,满足$n\le 500,1\leq m<998244353$
\begin{center}
	\begin{tabular}{|c|c|c|c|}
		\hline
		Subtask编号&分值&性质\\
		\hline
		1&15&$n,m\le5$\\
		\hline
		2&5&$m=1$\\
		\hline
		3&15&$m=2$\\
		\hline
		4&15&保证给定的是一条链\\
		\hline
		10&50&\\
		\hline
	\end{tabular}	
\end{center}

\clearpage
\begin{center}
	\large{Horseless}
\end{center}
\paragraph{题目描述}
\paragraph{}众所周知,小\text{P}从未拥有过\textbf{码},但是善良的小\text{H}想送给小\text{P}$S$个\textbf{码}
\paragraph{}小\text{H}总共有$n$个码力自动机,但是购买$i$号\textbf{一个}码力自动机需要消耗$C_i$个\textbf{码},而码力自动机可以在\textbf{工作时}的每个时刻给小$H$提供$V_i$个\textbf{码}
\paragraph{}但是码力自动机需要小\text{H}使用他人脑$1024$位计算机快速计算,所以每\textbf{一个}时刻只能有\textbf{一个}码力自动机为其工作,而小\text{H}可以在任意时刻\textbf{瞬间}购买任意多个码力自动机
\paragraph{}现在是$0$时刻,小\text{H}希望可以尽快为小\text{P}准备$S$个\textbf{码},需要你告诉他最快时间
\paragraph{}保证存在$C_i=0$的码力自动机

\paragraph{输入格式}
\paragraph{}从\emph{horseless.in}中读入数据
\begin{itemize}
\item 第一行两个整数$n$,$S$
\item 接下来$n$行每行两个整数$V_i,C_i$
\end{itemize}
\paragraph{输出格式}
\paragraph{}输出到\emph{horseless.out}中
\begin{itemize}
	\item 一行一个整数表示最快时刻
\end{itemize}
\paragraph{样例输入}
\begin{lstlisting}
3 9
1 0
2 3
5 4
\end{lstlisting}
\paragraph{样例输出}
\begin{lstlisting}
6
\end{lstlisting}
\paragraph{子任务}
\paragraph{}对于$100\%$的数据,满足$n\le 2\times 10^5,S,C_i,V_i\le 10^{18}$
\begin{center}
	\begin{tabular}{|c|c|c|c|}
		\hline
		Subtask编号&分值&性质\\
		\hline
		1&15&$n,S\le 18$\\
		\hline
		2&15&$n,S\le 2000$\\
		\hline
		3&20&$S\le 10^6$\\
		\hline
		4&20&$S\le 10^{12}$\\
		\hline
		10&30&\\
		\hline
	\end{tabular}	
\end{center}
\end{document}