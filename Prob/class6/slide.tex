\documentclass[9pt]{beamer}
\usepackage{xeCJK}
\usepackage[space,space,hyperref]{ctex}
\setsansfont{Ubuntu Mono}
\usefonttheme[onlymath]{serif}
\setCJKmonofont{AR PL UKai CN}
\usetheme{metropolis}
\title{水题选讲}
\date{\today}
\author{ljfcnyali}
\begin{document}
  \maketitle
  \begin{frame}{CF Gym 101190I}
    \par{}这是一道交互题
    \par{}现在有一张有向图,保证所有点出度均为$m$,但并不清楚点数和具体边集,你位于未知的某点,你需要遍历所有边集(可以重复经过)。
    \par{}每个点有一个石头,初始状态为中,你每次到达一个点时,可以修改石头的状态为左或右(但不能为中),同时,石头可以标记一条出边,初始状态为任意的一条出边。
    \par{}每次你可以将当前的石头标记的边顺时针转$x$条边,并且可以调整石头状态,再选择调整前石头标记的边顺时针转$y$条边的出边移动。
    \par{}注意,一个点的入边个数不清楚也不会在当前点观测到,同时保证原图强连通且所有出边相对顺序从未改变,每次交互库给出当前点石头状态左中右三种,你输出$x,L/R,y$表示一次移动。
    \par{}$m\le 20$且可以假设点数$\le 20$,允许重边自环,且交互次数$\le 20000$
  \end{frame}
  \begin{frame}{CF Gym 101190I}
    可以发现,石头指向的边一定为上次遇到改点时的出边,否则该信息没有用处。
    \pause
    
    很明显我们需要依次将所有点的出边全部走完,不妨假设我们现在需要将我们当前所在的点的所有出边走完,且我们按照某种DFS顺序处理,那么还未走完的点一定组成一条链。
    \pause
    
    那么我们假设所有出边走完的点状态为$L$,未走完的为$R$,未访问过的为$C$,且当前点在一条$R$组成的链上。
    \pause

    我们将当前点指向上次访问的边,依次处理其所有出边重复$m$次,并且每次我们试图回到当前点。
    \pause

    考虑如果当前出边走到一个$R$点,即形成了一个$R$环,那么标记此点为$L$,而我们可以遍历整个环知道该环大小,再将$L$修改回$R$且走环大小-1步回到正在处理的点
    \pause

    否则走到一个$L$点,那么$L$点一直按照其标记的点走一定会回到某一个$R$点上,再一直走即可以得到环大小,同理可以回到正在处理的点
    \pause

    如果走到的是$C$点,拓展$R$链即可
  \end{frame}
  \begin{frame}
    现在考虑当前处理的点所有出边访问完时,我们需要回到上一个$R$链上的点
    \pause

    若我们可以记录某一条出边组成的环,那么我们顺着那个环走即可回到上一个点
    \pause

    因为存在自环或$L$链与当前点组成的环,所以并不是所有环均可以回到上一个点,可以试图记录一个经过$R$点个数最多的环,这样的环一定合法
    \pause

    具体的,开一个栈记录$R$链上的所有点的信息,每次顺带维护以上信息即可。
  \end{frame}

\end{document}
