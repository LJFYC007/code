\documentclass[UTF8]{ctexart}
\usepackage{booktabs}
\usepackage{amsmath}
\usepackage{geometry}
\usepackage{fancyhdr}
\usepackage{listings}

\begin{document}
\title{solution}
\author{ljfcnyali}
\date{2019年8月3日}
\maketitle

\clearpage

\begin{center}
    \large{pythagorean}
\end{center}
\paragraph{$19\mathrm{pts}$}
\paragraph{}随便跑
\paragraph{$22\mathrm{pts}$}
\paragraph{}考虑枚举,时间复杂度$O(n^2)$
\paragraph{$58\mathrm{pts}$}
\paragraph{}首先预处理出$c\leq 5000$的所有答案,记$ans[i]$表示$c=i$的方案数,前缀和记录一下,$O(1)$询问即可
\paragraph{$1\mathrm{pts}$}
\paragraph{}这个有一个结论(可以参考洛谷日报中的一篇文章),这里讲一下结论吧:假设$a^2+b^2=c^2$则有$a=st,b=\frac{s^2-t^2}{2},c=\frac{s^2+t^2}{2}$并满足$1\leq t<s$($s,t$互质且均为奇数)
\paragraph{}所以可以枚举$c,t(1\leq t\leq \sqrt{c})$,时间复杂度为$O((r-l)\sqrt{r})$

\clearpage

\begin{center}
    \large{road}
\end{center}
\paragraph{}很明显是搜索(\sout{网络流}),显然从$1$号点开始爆搜就有分了,一般搜索状态为$(u,sum,cost)$表示当前$u$号点,总长度为$sum$,总快乐值为$cost$
\paragraph{}数据极其有梯度,考虑剪枝
\paragraph{}一、记录一个$ans$表示答案,如果$ans\leq sum$则退出
\paragraph{}二、同理如果$cost>k$则退出
\paragraph{}三、我们可以用$SPFA$跑一个从$n$号点的$dis1[i]$表示从$i$到$n$的最小快乐值的和,$dis2[i]$表示从$i$到$n$的最短距离,所以$cost+dis1[u]>k$或者$ans\leq sum+dis2[u]$都要退出
\paragraph{}四、记一个$use[u][cost][num]$表示当前在$u$总快乐值为$cost$总共经过$num$个点的最短总长度,如果当前$sum$大于则退出
\paragraph{}五、上面这个剪枝是错的(可以仔细思考一下),但加上这个剪枝后速度有很大提升,所以我们添加一个$rand$来控制几率,在部分几率的情况下不退出(我设置为$\frac{2}{5}$)

\clearpage

\begin{center}
    \large{water}
\end{center}
\paragraph{$13\mathrm{pts}$}
\paragraph{}$O(2^n)$搜索,对于每台空调判断建或不建排水箱
\paragraph{$19\mathrm{pts}$}
\paragraph{}可以考虑$DP$,设$dp[i]$表示当前考虑第$i$台空调并在此修建排水箱的最小总代价,可以得到转移方程$dp[i]=min(dp[j]+\sum_{k=j+1}^{i}P[k]*(X[i]-X[k])+C[i])(0\leq j<i)$,时间复杂度$O(n^3)$
\paragraph{$36\mathrm{pts}$}
\paragraph{}前缀和优化,时间复杂度$O(n^2)$
\paragraph{$32\mathrm{pts}$}
\paragraph{}设$sum1[i]=\sum_{j=1}^{i}P[j],sum2[i]=\sum_{j=1}^{i}X[j]*P[j]$,我们可以将方程化为$dp[i]=min(dp[j]+X[i]*(sum1[i]-sum1[j])-sum2[i]+sum2[j]+c[i])$
\paragraph{}化为$y=kx+b$的形式即$dp[i]+sum2[i]=X[i]*sum1[i]-X[i]*sum1[j]+sum2[j]+c[i]$,所以直接斜率优化即可
\end{document}