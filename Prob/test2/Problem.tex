\documentclass[UTF8]{ctexart}
\usepackage{booktabs}
\usepackage{amsmath}
\usepackage{geometry}
\usepackage{fancyhdr}
\usepackage{listings}

\begin{document}
\title{水题欢乐赛Day1}
\author{ljfcnyali}
\date{2019年8月3日}
\maketitle

\begin{table}[!htbp]
	\centering
	\begin{tabular}{|c|c|c|c|}
		\hline
		题目名称&签到题&道路&水题\\
		\hline
        题目类型&传统型&传统型&传统型\\
        \hline
        目录&pythagorean&road&water\\
        \hline
        可执行文件名&pythagorean&road&water\\
        \hline
        输入文件名&pythagorean.in&road.in&water.in\\
        \hline
        输出文件名&pythagorean.out&road.out&water.out\\
        \hline
        每个测试点时限&2s&1s&1s\\
        \hline
        内存限制&128MB&128MB&128MB\\
        \hline
        测试点/包数目&4&5&4\\
        \hline
        测试点是否等分&否&否&否\\
		\hline
    \end{tabular}
\end{table}

注意事项:

1.该比赛为\sout{水题欢乐赛}

2.如果你已经AK,请不要声张

\clearpage

\begin{center}
    \large{签到题(pythagorean)}
\end{center}
\paragraph{题目描述}
\paragraph{}众所周知满足$a^2+b^2=c^2$的数$(a,b,c)$($a,b,c$均为正整数)称之为勾股数,也称毕氏数
\paragraph{}现在我们需要求出$c$位于$[l,r]$区间中且满足$a^2+b^2=c^2$的本质不同的$(a,b,c)$数对个数
\paragraph{}本质不同指如果有两个数对$(a,b,c)$和$(xa,xb,xc)$均为勾股数($x$为正整数),则我们认为这两个数对是相同的
\paragraph{输入格式}
\paragraph{}有多组数据,第一行一个整数$T$表示数据组数
\paragraph{}对于每组数据,一行两个整数$l,r$表示$c$的范围为$[l,r]$
\paragraph{输出格式}
\paragraph{}共$T$行,每行一个整数表示第$i$组数据的答案
\paragraph{样例1输入}
\begin{lstlisting}
    1
    1 13
\end{lstlisting}
\paragraph{样例1输出}
\begin{lstlisting}
    2
\end{lstlisting}

\clearpage

\paragraph{数据范围}
\paragraph{}Subtask1:$19'$满足$0\leq l,r\leq 500$,$T=1$
\paragraph{}Subtask2:$22'$满足$0\leq l,r\leq 5000$,$T=1$
\paragraph{}Subtask3:$58'$满足$0\leq l,r\leq 5000$,$T\leq 100000$
\paragraph{}Subtask4:$1'$满足$0\leq l,r\leq 10^7$且$r-l\leq 50000$,$T=1$

\clearpage


\begin{center}
    \large{道路(road)}
\end{center}
\paragraph{题目描述}
\paragraph{}给定$n$个点$m$条无向边的图,对于每条从$u_i$到$v_i$的边,有长度$w_i$和快乐值$c_i$
\paragraph{}我们要从$1$到$n$,并且希望经过的所有路径的快乐值的和为$k$,同时满足总长度最短,询问最短的总长度。特别的,每个点最多经过一次
\paragraph{输入格式}
\paragraph{}第一行三个整数$n,m,k$,接下来$m$行,每行$u_i,v_i,w_i,c_i$,意义如题所示
\paragraph{输出格式}
\paragraph{}一行一个整数,表示最短的总长度,题目保证有解
\paragraph{样例1输入}
\begin{lstlisting}
    5 9 7
    1 2 5 2
    2 4 3 4
    4 5 2 1
    2 3 2 1
    3 1 2 4
    5 1 9 20
    1 2 3 2
    2 5 3 9
    4 4 2 1
\end{lstlisting}
\paragraph{样例1输出}
\begin{lstlisting}
    8
\end{lstlisting}

\clearpage

\paragraph{数据范围}
\paragraph{}Subtask1:$17'$满足$n=5,m=100$
\paragraph{}Subtask2:$12'$满足$n=8,m=160$
\paragraph{}Subtask3:$22'$满足$n=15,m=300$
\paragraph{}Subtask4:$16'$满足$n=17,m=350$
\paragraph{}Subtask5:$33'$满足$n=20,m=400$
\paragraph{}对于$100\%$的数据,满足$k\leq 100,w_i\leq 10,c_i\leq 10$,数据保证有解

\clearpage

\begin{center}
    \large{水题(water)}
\end{center}
\paragraph{题目描述}
\paragraph{}这是一道大水题
\paragraph{}有$n$台空调,我们假设第$1$台空调位于$0$,第$i(2\leq i\leq n)$台空调位于$X_i$(即所有空调位于一条直线上,该直线可以视为一条数轴),显而易见,开空调是需要排水的,我们假设空调$i$的排水量是$P_i$,排水箱当且仅当只可以修建在空调所处的位置,修建排水箱的代价为$C_i$,排水管道只可以向后修建(即水不可以向数轴负方向运送),空调$i$排水的代价为$dis*P_i$,其中$dis$表示$i$到排水箱的距离
\paragraph{}现在需要求出完成所有空调排水任务的最小总代价(运送代价+修建代价)
\paragraph{输入格式}
\paragraph{}第一行一个数$n$,接下来$n$行,每行$X_i,P_i,C_i$(保证$X_i$从小到大严格递增)
\paragraph{输出格式}
\paragraph{}一行一个数,表示最小总代价
\paragraph{样例1输入}
\begin{lstlisting}
    4
    0 2 3
    5 4 2
    7 3 2
    9 1 2
\end{lstlisting}
\paragraph{样例1输出}
\begin{lstlisting}
    9
\end{lstlisting}

\clearpage

\paragraph{数据范围}
\paragraph{}Subtask1:$13'$满足$n\leq 10$
\paragraph{}Subtask2:$19'$满足$n\leq 3*10^2$
\paragraph{}Subtask3:$36'$满足$n\leq 2*10^3$
\paragraph{}Subtask4:$32'$满足$n\leq 2*10^6$
\paragraph{}对于$100\%$的数据,满足$X_i,P_i,C_i\leq 2*10^9$

\end{document}